\chapter{The world}


 Much of the adventuring life is spent in dusty, forgotten tombs or in places of terror and life-threatening danger. It's commonplace to awaken from a short and fitful rest still deep in the belly of the world and surrounded by foes. When the time comes to emerge from these places--whether laden with the spoils of battle or beaten and bloody--an adventurer seeks out safety and solace.


 These are the comforts of civilization: a warm bath, a meal of mead and bread, the company of fellow men and elves and dwarves and halflings. Often thoughts of returning to these places are all that keep an adventurer from giving up altogether. All fight for gold and glory but who doesn't ache for a place to spend that gold and laugh around a fire, listening to tales of folly and adventure?


 This chapter covers the wider world--the grand and sweeping scope outside the dungeon. The always marching movement of the GM's fronts will shape the world and, in turn, the world reflects the actions the players take to stop or redirect them.
\subsection{Steadings}


 We call all the assorted communities, holds, and so on where there's a place to stay and some modicum of civilization \textbf{steadings}
, as in ``homestead.'' Steadings are places with at least a handful of inhabitants, usually humans, and some stable structures. They can be as big as a capital city or as small as few ramshackle buildings.
\section*{Creating the world}


 Remember how you started the first session? With action either underway or impending? At some point the characters are going to need to retreat from that action, either to heal their wounds or to celebrate and resupply.


 When the players leave the site of their first adventure for the safety of civilization it's time to start drawing the campaign map. Take a large sheet of paper (plain white if you like or hex-gridded if you want to get fancy), place it where everyone can see, and make a mark for the site of the adventure. Use pencil: this map will change. It can be to-scale and detailed or broad and abstract, depending on your preference, just make it obvious. Keep the mark small and somewhere around the center of the paper so you have space to grow.


 Now add the nearest steading, a place the characters can go to rest and gather supplies. Draw a mark for that place on the map and fill in the space between with some terrain features. Try to keep it within a day or two of the site of their first adventure--a short trip through a rocky pass or some heavy woods is suitable, or a wider distance by road or across open ground.


 When you have time (after the first session or during a snack break, for example) use the rules to create the first steading. Consider adding marks for other places that have been mentioned so far, either details from character creation or the steading rules themselves.
\section*{While You're In Town \ldots }


 When the players visit a steading there are some special moves they'll be able to make. These still follow the fictional flow of the game. When the players arrive, ask them ``What do you do?'' The players' actions will, more often than not, trigger a move from this list. They cover respite, reinvigoration, and resupply--opportunities for the players to gather their wits and spend their treasure. Remember that a steading isn't a break from reality. You're still making hard moves when necessary and thinking about how the players' actions (or inaction) advances your fronts. The impending doom is always there, whether the players are fighting it in the dungeon or ignoring it while getting drunk in the local tavern.


 Don't let a visit to a steading become a permanent respite. Remember, Dungeon World is a scary, dangerous place. If the players choose to ignore that, they're giving you a golden opportunity to make a hard move. Fill the characters' lives with adventure whether they're out seeking it or not. These moves exist so you can make a visit to town an interesting event without spending a whole session haggling over the cost of a new baldric.
\section*{Elements of a Steading}


 A steading is any bit of civilization that offers some amount of safety to its inhabitants. Villages, towns, keeps, and cities are the most common steadings. Steadings are described by their tags. All steadings have tags indicating prosperity, population, and defenses. Many will have tags to illustrate their more unusual properties.


 Steadings are differentiated based on size. The size indicates roughly how many people the steading can support. The population tag tells you if the current population is more than or less than this amount.


 Villages are the smallest steadings. They're usually out of the way, off the main roads. If they're lucky they can muster some defense but it's often just rabble with pitchforks and torches. A village stands near some easily exploitable resource: rich soil, plentiful fish, an old forest, or a mine. There might be a store of some sort but more likely its people trade among themselves. Coin is scarce.


 Towns have a hundred or so inhabitants. They're the kind of place that springs up around a mill, trading post, or inn and usually have fields, farms, and livestock of some kind. They might have a standing militia of farmers strong enough to wield a blade or shoot a bow. Towns have the basics for sale but certainly no special goods. Usually they'll focus on a local product or two and do some trade with travelers.


 A keep is a steading built specifically for defense--sometimes of a particularly important location like a river delta or a rich gold mine. Keeps are found at the frontier edges of civilization. Inhabitants are inured to the day-to-day dangers of the road. They're tough folks that number between a hundred and a thousand, depending on the size of the keep and the place it defends. Keeps won't often have much beyond their own supplies, traded to them from nearby villages, but will almost always have arms and armor and sometimes a rare magical item found in the local wilds.


 From bustling trade center to sprawling metropolis, the city represents the largest sort of steading in Dungeon World. These are places where folk of many races and kinds can be found. They often exist at the confluence of a handful of trade routes or are built in a place of spiritual significance. They don't often generate their own raw materials for trade, relying on supplies from villages nearby for food and resources, but will always have crafted goods and some stranger things for sale to those willing to seek them.


 Prosperity indicates what kinds of items are usually available. Population indicates the number of inhabitants relative to the current size of the steading. Defenses indicate the general scope of arms the steading has. Tags in these categories can be adjusted. -Category means to change the steading to the next lower tag for that category (so Moderate would become Poor with -Prosperity). +Category means to change the steading to the next higher tag (so Shrinking becomes Steady with +Population). Tags in those categories can also be compared like numbers. Treat the lowest tag in that category as 1 and each successive tag as the next number (so Dirt is 1, Poor is 2, etc.).


 Tags will change over the course of play. Creating a steading provides a snapshot of what that place looks like \emph{right now}. As the players spend time in it and your fronts progress the world will change and your steadings with it.
\subsection{Adding Steadings}


 You add your first steading when you create the campaign map--it's the place the players go to rest and recover. When you first draw it on the map all you need is a name and a location.


 When you have the time you'll use the rules below to create the steading. The first steading is usually a village, but you can use a town if the first adventure was closely tied to humans (for example, if the players fought a human cult). Create it using the rules below.


 Once you've created the first steading you can add other places referenced in its tags (the oath, trade, and enmity tags in particular) or anywhere else that's been referred to in play. Don't add too much in the first session, leave blanks and places to explore.


 As play progresses the characters will discover new locales and places of interest either directly, by stumbling upon them in the wild, or indirectly, by hearing about them in rumors or tales. Add new steadings, dungeons, and other locations to the map as they're discovered or heard about. Villages are often near a useful resource. Towns are often found at the point where several villages meet to trade. Keeps watch over important locations. Cities rely on the trade and support of smaller steads. Dungeons can be found anywhere and in many forms.


 Whenever you add a new steading use the rules to decide its tags. Consider adding a distinctive feature somewhere nearby. Maybe a forest, some old standing stones, an abandoned castle, or whatever else catches your fancy or makes sense. A map of only steadings and ruins with nothing in between is dull; don't neglect the other features of the world.
\subsection{Steading Tags}
\subsubsection{Prosperity}


 \emph{Dirt}
: Nothing for sale, nobody has more than they need (and they're lucky if they have that). Unskilled labor is cheap.


 \emph{Poor}
: Only the bare necessities for sale. Weapons are scarce unless the steading is heavily defended or militant. Unskilled labor is readily available.


 \emph{Moderate}
: Most mundane items are available. Some types of skilled laborers.


 \emph{Wealthy}
: Any mundane item can be found for sale. Most kinds of skilled laborers are available, but demand is high for their time.


 \emph{Rich}
: Mundane items and more, if you know where to find them. Specialist labor available, but at high prices.
\subsubsection{Population}


 \emph{Exodus}
: The steading has lost its population and is on the verge of collapse.


 \emph{Shrinking}
: The population is less than it once was. Buildings stand empty.


 \emph{Steady}
: The population is in line with the current size of the steading. Some slow growth.


 \emph{Growing}
: More people than there are buildings.


 \emph{Booming}
: Resources are stretched thin trying to keep up with the number of people.
\subsubsection{Defenses}


 \emph{None}
: Clubs, torches, farming tools.


 \emph{Militia}
: There are able-bodied men and women with worn weapons ready to be called, but no standing force.


 \emph{Watch}
: There are a few watchers posted who look out for trouble and settle small problems, but their main role is to summon the militia.


 \emph{Guard}
: There are armed defenders at all times with a total pool of less than 100 (or equivalent). There is always at least one armed patrol about the steading.


 \emph{Garrison}
: There are armed defenders at all times with a total pool of 100--300 (or equivalent). There are multiple armed patrols at all times.


 \emph{Battalion}
: As many as 1,000 armed defenders (or equivalent). The steading has manned maintained defenses as well.


 \emph{Legion}
: The steading is defended by thousands of armed soldiers (or equivalent). The steading's defenses are intimidating.
\subsubsection{Other Tags}


 \emph{Safe}
: Outside trouble doesn't come here until the players bring it. Idyllic and often hidden, if the steading would lose or degrade another beneficial tag get rid of safe instead.


 \emph{Religion}
: The listed deity is revered here.


 \emph{Exotic}
: There are goods and services available here that aren't available anywhere else nearby. List them.


 \emph{Resource}
: The steading has easy access to the listed resource (e.g., a spice, a type of ore, fish, grapes). That resource is significantly cheaper.


 \emph{Need}
: The steading has an acute or ongoing need for the listed resource. That resource sells for considerably more.


 \emph{Oath}
: The steading has sworn oaths to the listed steadings. These oaths are generally of fealty or support, but may be more specific.


 \emph{Trade}
: The steading regularly trades with the listed steadings.


 \emph{Market}
: Everyone comes here to trade. On any given day the available items may be far beyond their prosperity. +1 to supply.


 \emph{Enmity}
: The steading holds a grudge against the listed steadings.


 \emph{History}
: Something important once happened here, choose one and detail or make up your own: battle, miracle, myth, romance, tragedy.


 \emph{Arcane}
: Someone in town can cast arcane spells for a price. This tends to draw more arcane casters, +1 to recruit when you put out word you're looking for an adept.


 \emph{Divine}
: There is a major religious presence, maybe a cathedral or monastery. They can heal and maybe even raise the dead for a donation or resolution of a quest. Take +1 to recruit priests here.


 \emph{Guild}
: The listed type of guild has a major presence (and usually a fair amount of influence). If the guild is closely associated with a type of hireling, +1 to recruit that type of hireling.


 \emph{Personage}
: There's a notable person who makes their home here. Give them a name and a short note on why they're notable.


 \emph{Dwarven}
: The steading is significantly or entirely dwarves. Dwarven goods are more common and less expensive than they typically are.


 \emph{Elven}
: The steading is significantly or entirely elves. Elven goods are more common and less expensive than they typically are.


 \emph{Craft}
: The steading is known for excellence in the listed craft. Items of their chosen craft are more readily available here or of higher quality than found elsewhere. 


 \emph{Lawless}
: Crime is rampant; authority is weak.


 \emph{Blight}
: The steading has a recurring problem, usually a type of monster.


 \emph{Power}
: The steading holds sway of some type. Typically political, divine, or arcane.
\subsection{Steading Names}


 Graybark, Nook's Crossing, Tanner's Ford, Goldenfield, Barrowbridge, Rum River, Brindenburg, Shambles, Covaner, Enfield, Crystal Falls, Castle Daunting, Nulty's Harbor, Castonshire, Cornwood, Irongate, Mayhill, Pigton, Crosses, Battlemoore, Torsea, Curland, Snowcalm, Seawall, Varlosh, Terminum, Avonia, Bucksburg, Settledown, Goblinjaw, Hammerford, Pit, The Gray Fast, Ennet Bend, Harrison's Hold, Fortress Andwynne, Blackstone
\section*{Making a Village}


 By default a village is Poor, Steady, Militia, Resource (your choice) and has an Oath to another steading of your choice. If the village is part of a kingdom or empire choose one:
\begin{itemize}
\item The village is somewhere naturally defended: Safe, -Defenses
\item The village has abundant resources that sustain it: +Prosperity, Resource (your choice), Enmity (your choice)
\item The village is under the protection of another steading: Oath (that steading), +Defenses
\item The village is on a major road: Trade (your choice), +Prosperity
\item The village is built around a wizard's tower: Personage (the wizard), Blight (arcane creatures)
\item The village was built on the site of religious significance: Divine, History (your choice)

\end{itemize}


 Choose one problem:
\begin{itemize}
\item The village is in arid or uncultivable land: Need (Food)
\item The village is dedicated to a deity: Religious (that deity), Enmity (a settlement of another deity)
\item The village has recently fought a battle: -Population, -Prosperity if they fought to the end, -Defenses if they lost.
\item The village has a monster problem: Blight (that monster), Need (adventurers)
\item The village has absorbed another village: +Population, Lawless
\item The village is remote or unwelcoming: -Prosperity, Dwarven or Elven

\end{itemize}
\section*{Making a Town}


 By default a town is Moderate, Steady, Watch, and Trade (two of your choice). If the town is listed as Trade by another steading choose one:
\begin{itemize}
\item The town is booming: Booming, Lawless
\item The town stands on a crossroads: Market, +Prosperity
\item The town is defended by another steading: Oath (that steading), +Defenses
\item The town is built around a church: Power (Divine)
\item The town is built around a craft: Craft (your choice), Resource (something required for that craft)
\item The town is built around a military post: +Defenses

\end{itemize}


 Choose one problem:
\begin{itemize}
\item The town has grown too big for an important supply (like grain, wood, or stone): Need (that resource), Trade (a village or town with that resource)
\item The town offers defense to others: Oath (your choice), -Defenses
\item The town is notorious for an outlaw who is rumored to live there: Personage (the outlaw), Enmity (where the crimes were committed)
\item The town has cornered the market on a good or service: Exotic (that good or service), Enmity (a settlement with ambition)
\item The town has a disease: -Population
\item The town is a popular meeting place: +Population, Lawless

\end{itemize}
\section*{Making a Keep}


 By default a keep is Poor, Shrinking, Guard, Need (Supplies), Trade (someplace with supplies), Oath (your choice). If the keep is owed fealty by at least one settlement choose one:
\begin{itemize}
\item The keep belongs to a noble family: +Prosperity, Power (Political)
\item The keep is run by a skilled commander: Personage (the commander), +Defenses
\item The keep stands watch over a trade road: +Prosperity, Guild (trade)
\item The keep is used to train special troops: Arcane, -Population
\item The keep is surrounded by fertile land: remove Need (Supplies)
\item The keep stands on a border: +Defenses, Enmity (steading on the other side of the border)

\end{itemize}


 Choose one problem
\begin{itemize}
\item The keep is built on a naturally defensible position: Safe, -Population
\item The keep was a conquest from another power: Enmity (steadings of that power)
\item The keep is a safe haven for brigands: Lawless
\item The keep was built to defend from a specific threat: Blight (that threat)
\item The keep has seen horrible bloody war: History (Battle), Blight (Restless Spirits)
\item The keep is given the worst of the worst: Need (Skilled Recruits)

\end{itemize}
\section*{Making a City}


 By default a city is Moderate, Steady, Guard, Market, and Guild (one of your choice). It also has Oaths with at least two other steadings, usually a town and a keep. If the city has trade with at least one steading and fealty from at least one steading choose one:
\begin{itemize}
\item The city has permanent defenses, like walls: +Defenses, Oath (your choice)
\item The city is ruled by a single individual: Personage (the ruler), Power (Political)
\item The city is diverse: Dwarven or Elven or both
\item The city is a trade hub: Trade (every steading nearby), +Prosperity
\item The city is ancient, built on top of its own ruins: History (your choice), Divine
\item The city is a center of learning: Arcane, Craft (your choice), Power (Arcane)

\end{itemize}


 Choose one problem:
\begin{itemize}
\item The city has outgrown its resources: +Population, Need (food)
\item The city has designs on nearby territory: Enmity (nearby steadings), +Defenses
\item The city is ruled by a theocracy: -Defenses, Power (Divine)
\item The city is ruled by the people: -Defenses, +Population
\item The city has supernatural defenses: +Defenses, Blight (related supernatural creatures)
\item The city lies on a place of power: Arcane, Personage (whoever watches the place of power), Blight (arcane creatures)

\end{itemize}
\section*{Fronts on the Campaign Map}


 Your steadings are not the only thing on the campaign map. In addition to steadings and the areas around them your fronts will appear on the map, albeit indirectly.


 Fronts are organizational tools, not something the characters think of, so don't put them on the map directly. The orcs of Olg'gothal may be a front but don't just draw them on the map. Instead for each front add some feature to the map that indicates the front's presence. You can label it if you like, but use the name that the characters would use, not the name you gave the front.


 For example, the orcs of Olg'gothal could be marked on the map with a burning village they left behind, fires in the distance at night, or a stream of refugees. Lord Xothal, a lich, might be marked by the tower where dead plants take root and grow.


 As your fronts change, change the map. If the players cleanse Xothal's tower redraw it. If the orcs are driven off erase the crowds of refugees.
\section*{Updating the Campaign Map}


 The campaign map is updated between sessions or whenever the players spend significant downtime in a safe place. Updates are both prescriptive and descriptive: if an event transpires that, say, gathers a larger fighting force to a village, update the tags to reflect that. Likewise if a change in tags mean that a village has a bigger fighting force you'll likely see more armored men in the street.


 Between each session check each of the conditions below. Go down the list and check each condition for all steadings before moving to the next. If a condition applies, apply its effects.
\subsubsection{Growth}


 When \textbf{a village or town is booming and its prosperity is above moderate}
 you may reduce prosperity and defenses to move to the next largest type. New towns immediately gain market and new cities immediately gain guild (your choice).
\subsubsection{Collapse}


 When \textbf{a steading's population is in exodus and its prosperity is poor or less}
 it shrinks. A city becomes a town with a steady population and +prosperity. A keep becomes a town with +defenses and a steady population. A town becomes a village with steady population and +prosperity. A village becomes a ghost town.
\subsubsection{Want}


 When \textbf{a steading has a need that is not fulfilled}
 (through trade, capture, or otherwise) that steading is in want. It gets either -prosperity, -population, or loses a tag based on that resource like craft or trade, your choice.
\subsubsection{Trade}


 When \textbf{trade is blocked}
 because the source of that trade is gone, the route is endangered, or political reasons, the steading has a choice: gain need (a traded good) or take -prosperity.
\subsubsection{Capture}


 When \textbf{control of a resource changes}
 remove that resource from the tags of the previous owner and add it to the tags of the new owner (if applicable). If the previous owner has a craft or trade based on that resource they now have need (that resource). If the new owner had a need for that resource, remove it.
\subsubsection{Profit}


 When \textbf{a steading has more trade than its current prosperity}
 it gets +prosperity.
\subsubsection{Surplus}


 When \textbf{a steading has a resource that another steading needs}
 unless enmity or other diplomatic reasons prevent it they set up trade. The steading with the resource gets +prosperity and their choice of oaths, +population, or +defenses; the steading with the need erases that need and adds trade.
\subsubsection{Aid}


 When \textbf{a steading has oaths to a steading under attack}
 that steading may take -defenses to give the steading under attack +defenses.
\subsubsection{Embattled}


 When \textbf{a steading is surrounded by enemy forces}
 it suffers losses. If it fights back with force it gets -defenses. If its new defenses are watch or less it also gets -prosperity. If it instead tries to wait out the attack it gets -population. If its new population is shrinking or less it loses a tag of your choice. If the steading's defenses outclass the attacker's (your call if it's not clear, or make it part of an adventure front) the steading is no longer surrounded.
\subsubsection{Opportunity}


 When \textbf{a steading has enmity against a weaker steading}
 they may attack. Subtract the distance (in rations) between the steadings from the steading with enmity's defenses. If the result is greater than the other steading's defenses +defense for each step of size difference (village to town, town to keep, keep to city) they definitely attack. Otherwise it's your call: has anything happened recently to stoke their anger? The forces of the attacker embattle the defender, while they maintain the attack they're -defenses.
\subsubsection{Clash}


 When \textbf{two steadings both attack each other}
 their forces meet somewhere between them and fight. If they're evenly matched they both get -defenses and their troops return home. If one has the advantage they take -defenses while the other takes -2 defenses.
\subsection{Other Updates}


 The conditions above detail the most basic of interactions between steadings, of course the presence of your fronts and the players mean things can get far more complex. Since tags are descriptive, add them as needed to reflect the players' actions and your fronts' effects on the world.


