\section{Folk of the Realm}

\HRule
\monster{Acolyte}{}
\index{Monsters by Name!Acolyte}

\HRule
``Can't all be the High Priest, they said. Can't all wield the White Spire, they said. Scrub the floor, they told me. The Cthonic Overgod don't want a messy floor, do he? They said it'd be enlightenment and magic. Feh. It's bruised knees and dishpan hands. If only I'd been a cleric, instead.'' \emph{Instinct}: To serve dutifully
\begin{itemize}
\item Follow dogma
\item Offer eternal reward for mortal deeds
\end{itemize}

\HRule
\monster{Adventurer}{Horde, Intelligent}
\index{Monsters by Name!Adventurer}

Sword (d6 damage)\hspace*{\fill} 3 HP 1 Armor

\emph{Close}

\textbf{Special Qualities:}
Endless enthusiasm

\HRule
``Scum of the earth, they are. A troupe of armored men and women come sauntering into town, brandishing what, for all intents and purposes, is enough magical and mundane power to level the whole place. Bringing with them bags and bags of loot, still dripping blood from whatever poor sod they had to kill to get it. An economical fiasco waiting to happen, if you ask me. The whole system becomes completely uprooted. Dangerous, unpredictable murder-hobos. Oh, wait, you're an adventurer? I take it all back.'' \emph{Instinct}: To adventure or die trying
\begin{itemize}
\item Go on a fool's errand
\item Act impulsively
\item Share tales of past exploits
\end{itemize}
\newpage
\HRule
\monster{Bandit}{Horde, Intelligent, Organized}
\index{Monsters by Name!Bandit}

Dirk (d6 damage)\hspace*{\fill} 3 HP 1 Armor

\emph{Close}

\HRule
Desperation is the watchword of banditry. When times are tough, what else is there to do but scavenge a weapon and take up with a clan of nasty men and women? Highway robbery, poaching, scams and cons and murder most foul but we've all got to eat so who can blame them? Then again, there's evil in the hearts of some and who's to say that desperation isn't a need to sate one's baser lusts? Anyway--it's this or starve, sometimes. \emph{Instinct}: To rob
\begin{itemize}
\item Steal something
\item Demand tribute
\end{itemize}

\HRule
\monster{Bandit King}{Solitary, Intelligent, Organized}
\index{Monsters by Name!Bandit King}

Trusty knife (b[2d10] damage)\hspace*{\fill} 12 HP 1 Armor

\emph{Close}

\HRule
Better to rule in hell than serve in heaven. \emph{Instinct}: To lead
\begin{itemize}
\item Make a demand
\item Extort
\item Topple power
\end{itemize}

\HRule
\monster{Fool}{}
\index{Monsters by Name!Fool}

\HRule
There's not but one person in all the King's court allowed to speak the truth. The real, straight-and-honest truth about anything. The fool couches it all in bells and prancing and chalky face-paint, but who else gets to tell the King what's what? You can trust a fool, they say, especially when he's made you red-faced and you'd just as soon drown him in a cesspit. \emph{Instinct}: To mock
\begin{itemize}
\item Expose injustice
\item Play a trick
\end{itemize}

\HRule
\monster{Guardsman}{Group, Intelligent, Organized}
\index{Monsters by Name!Guardsman}

Spear (d8 damage)\hspace*{\fill} 6 HP 1 Armor

\emph{Close, Reach}

\HRule
Noble protector or merely drunken lout, it often makes no difference to these sorts. Falling shy of a noble knight, the proud town guard is an ancient profession nonetheless. These folks of the constabulary often dress in the colors of their lord (when you can see it under the mud) and, depending on the richness of that lord, might even have a decent weapon and some armor that fits. Those are the lucky ones. Even so, someone has to be there to keep an eye on the gate when the Black Riders have been spotted in the woods. Too many of us owe our lives to these souls--remember that the next time one is drunkenly insulting your mother, hmm? \emph{Instinct}: To do as ordered
\begin{itemize}
\item Uphold the law
\item Make a profit
\end{itemize}

\HRule
\monster{Halfling Thief}{Solitary, Small, Intelligent, Stealthy, Devious}
\index{Monsters by Name!Halfling Thief}

Dagger (w[2d8] damage)\hspace*{\fill} 12 HP 1 Armor

\emph{Close}

\HRule
It would be foolish, now, to draw conclusions about folks just because they happen to be good at one thing or another. Then again, a spade's a spade, isn't it. Or maybe just the goodly, soft-and-sweet type of Halfling have the mind to stay in their grassy-hill homes and aren't the type you find in the slums and taverns of the mannish world. Perhaps they're there to cut your purse for calling them ``halfling'' in the first place. Not all take so kindly to the title. Or they're playing a game, pretending to be a child in need of alms--and your arrogant eyes can't even see the difference until too late. Well, it matters little. They're gone with your coin before you even realize you deserved it. \emph{Instinct}: To live a life of stolen luxury
\begin{itemize}
\item Steal
\item Put on the appearance of friendship
\end{itemize}

\HRule
\monster{Hedge Wizard}{Magical}
\index{Monsters by Name!Hedge Wizard}

\HRule
Not all those who wield the arcane arts are adventuring wizards. Nor necromancers in mausoleums or sorcerers of ancient bloodline. Some are just old men and women, smart enough to have discovered a trick or two. It might make them a bit batty to come by that knowledge, but if you've a curse to break or a love to prove, might be that a hedge wizard will help you, if you can find his rotten hut in the swamp and pay the price he asks. \emph{Instinct}: To learn
\begin{itemize}
\item Cast almost the right spell (for a price)
\item Make deals beyond their ken
\end{itemize}

\HRule
\monster{High Priest}{}
\index{Monsters by Name!High Priest}

\HRule
Respected by all who gaze upon them, the high priests and abbesses of Dungeon World are treated with a sort of reverence. Whether they pay homage to Ur-thuu-hak, God of Swords, or whisper quiet prayers to Namiah, precious daughter of peace, they know a thing or two that you and I won't ever know. The gods speak to them as a hawker-of-wares might speak to us in the marketplace. For this, for the bearing-of-secrets and the knowing-of-things, we give them a wide berth as they pass in their shining robes. \emph{Instinct}: To lead
\begin{itemize}
\item Set down divine law
\item Reveal divine secrets
\item Commission divine undertakings
\end{itemize}
\newpage
\HRule
\monster{Hunter}{Group, Intelligent}
\index{Monsters by Name!Hunter}

Ragged bow (d6 damage)\hspace*{\fill} 6 HP 1 Armor

\emph{Near, Far}

\HRule
The wilds are home to more than just beasts of horn and scale. There are men and women out there, too--those who smell blood on the wind and stalk the plains in the skins of their prey. Whether with a trusty longbow bought on a rare trip into the city or with a knife of bone and sinew, these folk have more in common with the things they track and eat than with their own kind. Solemn, somber and quiet, they find a sort of peace in the wild. \emph{Instinct}: To survive
\begin{itemize}
\item Bring back news from the wilds
\item Slay a beast
\end{itemize}

\HRule
\monster{Knight}{Solitary, Intelligent, Organized, Cautious}
\index{Monsters by Name!Knight}

Sword (b[2d10] damage)\hspace*{\fill} 12 HP 4 Armor

\emph{Close}

\HRule
What youngster doesn't cling to the rail at the mighty joust, blinded by the sun on their glittering armor, wishing they could be the one adorned in steel and riding to please the King and Queen? What peasant youth with naught but a loaf of bread and a lame sow doesn't wish to trade it all in for the lance and the bright pennant? A knight is many things--a holy warrior, a sworn sword, a villain sometimes, too, but a knight cannot help but be a symbol to all who see her. A knight means something. \emph{Instinct}: To live by a code
\begin{itemize}
\item Make a moral stand
\item Lead soldiers into battle
\end{itemize}
\newpage
\HRule
\monster{Merchant}{} 
\index{Monsters by Name!Merchant}

\HRule
``Ten foot poles. Get your ten foot poles, here. Torches, bright and hot. Mules, too--stubborn but immaculately bred. Need a linen sack, do you? Right over here! Come and get your ten foot poles!'' \emph{Instinct}: To profit
\begin{itemize}
\item Propose a business venture
\item Offer a ``deal''
\end{itemize}

\HRule
\monster{Noble}{} 
\index{Monsters by Name!Noble}

\HRule
Are they granted their place by the gods, perhaps? Is that why they're able to pass their riches and power down by birth? Some trick or enchantment of the blood, maybe. The peasant bends his knee and scrapes and toils and the noble wears the finery of his place and, they say, we all have our burdens to bear. Seems to me that some of us have burdens of stone and some carry their weight in gold. It's a tough life. \emph{Instinct}: To rule
\begin{itemize}
\item Issue an order
\item Offer a reward
\end{itemize}

\HRule
\monster{Peasant}{} 
\index{Monsters by Name!Peasant}

\HRule
Covered in muck, downtrodden at the bottom of the great chain of being, we all stand on the backs of those who grow our food on their farms. Some peasants do better than others, but none will ever see a coin of gold in their day. They'll dream at night of how someday, somehow, they'll fight a dragon and save a princess. Don't act like you weren't one before you lost what little sense you had, adventurer. \emph{Instinct}: To get by
\begin{itemize}
\item Plead for help
\item Offer a simple reward and gratitude
\end{itemize}
\newpage
\HRule
\monster{Rebel}{Horde, Intelligent, Organized}
\index{Monsters by Name!Rebel}

Axe (d6 damage)\hspace*{\fill} 3 HP 1 Armor

\emph{Close}

\HRule
In the countryside they'd be called outlaw and driven off or killed. The city, though, is full of places to hide. Damp basements to pore over maps and to plan and plot against a corrupt system. Like rats, they gnaw away at order, either to supplant it anew or just erode the whole thing. The line between change and chaos is a fine one--some rebels walk that thin line and others just want to see it all torched. Disguise, a knife in the dark or a thrown torch at the right moment are all tools of the rebel. The burning brand of anarchy is a common fear amongst the nobles of Dungeon World. These men and women are why. \emph{Instinct}: To upset order
\begin{itemize}
\item Die for a cause
\item Inspire others
\end{itemize}

\HRule
\monster{Soldier}{Horde, Intelligent, Organized}
\index{Monsters by Name!Soldier}

Spear (d6 damage)\hspace*{\fill} 3 HP 1 Armor

\emph{Close, Reach}

\HRule
For a commoner with a strong arm, sometimes it's this or be a bandit. It's wear the colors and don ill-fitting armor and march into the unknown with a thousand other scared men and women conscripted to fight the wars of our time. They could be hiding out in the woods instead, living off poached elk and dodging the king's guard. Better to risk one's life in service to a cause. To bravely toss one's lot in with one's fellows and hope to come out the other side still in one piece. Besides, the nobles need strong men and women. What is it they say? A handful of soldiers beats a mouthful of arguments. \emph{Instinct}: To fight
\begin{itemize}
\item March into battle
\item Fight as one
\end{itemize}
\newpage
\HRule
\monster{Spy}{} 
\index{Monsters by Name!Spy}

\HRule
Beloved of kings but never truly trusted. Mysterious, secretive and alluring, the life of a spy is, if you ask a commoner, full of romance and intrigue. They're a knife in the dark and a pair of watchful eyes. A spy can be your best friend, your lover or that old man you see in the market every day. One never knows. Hells, maybe you're a spy--they say there's magic that can turn folks' minds without them ever knowing it. How can we trust you? \emph{Instinct}: To infiltrate
\begin{itemize}
\item Report the truth
\item Double cross
\end{itemize}

\HRule
\monster{Tinkerer}{} 
\index{Monsters by Name!Tinkerer}

\HRule
It's said that if you see a tinker on the road and you don't offer him a swig of ale or some of your food that he'll leave a curse of bad luck behind. A tinker is a funny thing. These strange folk often travel the roads between towns with their oddment carts and favorite mules. With a ratty dog and always a story to tell. Sometimes the mail, too, if you're lucky and live in a place where Queen's Post won't go. If you're kind, maybe they'll sell you a rose that never wilts or a clock that chimes with the sound of faerie laughter. Or maybe they're just antisocial peddlers. You never know, right? \emph{Instinct}: To create
\begin{itemize}
\item Offer an oddity at a price
\item Spin tales of great danger and reward in far-off lands
\end{itemize}
