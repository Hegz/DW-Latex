\chapter{Playing the Game}


 Playing Dungeon World means having a conversation; somebody says something, then you reply, maybe someone else chimes in. We talk about the fiction--the world of the characters and the things that happen around them. As we play, the rules will chime in, too. They have something to say about the world. There are no turns or rounds in Dungeon World, no rules to say whose turn it is to talk. Instead players take turns in the natural flow of the conversation, which always has some back-and-forth. The GM says something, the players respond. The players ask questions or make statements, the GM tells them what happens next. Dungeon World is never a monologue; it's always a conversation.


 The rules help shape the conversation of play. While the GM and the players are talking, the rules and the fiction are talking, too. Every rule has an explicit fictional trigger that tells you when it is meant to come into the conversation.


 Like any conversation, the time you spend listening is just as important as the time you spend talking. The details established by the other people at the table (the GM and the other players) are important to you: they might change what moves you can make, set up an opportunity for you, or create a challenge you have to face. The conversation works best when we all listen, ask questions, and build on each other's contributions.


 This chapter is all about how to play Dungeon World. Here, you'll find information about the rules--how they arise from and contribute to the game. We'll cover both general rules, like making moves, and more specific rules, like those for dealing with damage and hit points.
\section*{Ability Scores and Modifiers}

\index{Ability Scores}
 Many of the rules discussed in this chapter rely on a player character's abilities and their modifiers. The abilities are Strength, Constitution, Dexterity, Intelligence, Wisdom, and Charisma. They measure a player character's raw ability in each of those areas on a scale from 3 to 18, where 18 is the peak of mortal ability.


\index{Ability Modifiers}
 Each ability has a modifier associated with it that is used when rolling with that ability. These are written as three-letter abbreviations: Str, Con, Dex, Int, Wis, Cha. Modifiers range from -3 to +3. The modifier is always derived from the current ability score.
\section*{Making Moves}


 The most basic unit of rules in Dungeon World is the move. A move looks like this:

\HRule
 When you \textbf{attack an enemy in melee}, roll+Str. *On a 10+, you deal your damage to the enemy and avoid their attack. At your option, you may choose to do +1d6 damage but expose yourself to the enemy's attack. *On a 7--9, you deal your damage to the enemy and the enemy makes an attack against you.
\HRule

 Moves are rules that tell you when they trigger and what effect they have. A move depends on a fictional action and always has some fictional effect. ``Fictional'' means that the action and effect come from the world of the characters we're describing. In the move above the trigger is ``when you attack an enemy in melee.'' The effect is what follows: a roll to be made and differing fictional effects based on the outcome of the roll.


 When a player describes their character doing something that triggers a move, that move happens and its rules apply. If the move requires a roll, its description will tell you what dice to roll and how to read their results.


 A character can't take the fictional action that triggers a move without that move occurring. For example, if Isaac tells the GM that his character dashes past a crazed axe-wielding orc to the open door, he makes the defy danger move because its trigger is ``when you act despite an imminent threat.'' Isaac can't just describe his character running past the orc without making the defy danger move and he can't make the defy danger move without acting despite an imminent threat or suffering a calamity. The moves and the fiction go hand-in-hand.


 Everyone at the table should listen for when moves apply. If it's ever unclear if a move has been triggered, everyone should work together to clarify what's happening. Ask questions of everyone involved until everyone sees the situation the same way and then roll the dice, or don't, as the situation requires.


 The GM's monsters, NPCs, and other assorted beasties also have moves, but they work differently.
\subsection{Moves and Dice}


 Most moves include the phrase ``roll+x'' where ``x'' is one of your character's ability modifiers (CON for example). Unless the move tells you otherwise, that ``roll'' always means that you roll two six-sided dice and add their results to the value of the modifier. Some moves will have you add some other value to your roll instead of an ability modifier.

\begin{quote}
\emph{I'm making a move that asks me to roll+STR and my STR modifier is +1. I rolled two six-sided dice, got a one and a four. My total is six.}
\end{quote}


 The results always fall into three basic categories. A total of 10 or higher (written 10+) is the best outcome. A total of 7--9 is still a success but it comes with compromises or cost. A 6 or lower is trouble, but you also get to mark XP\@.


 Each move will tell you what happens on a 10+ and a 7--9. Most moves won't say what happens on a 6-, that's up to the GM but you also always mark XP\@.


 Damage rolls work a little differently. They use different dice depending on who or what is dealing damage.
\subsection{Moves and Equipment}
\index{Equipment!moves and}

 The most important role of a character's equipment is to help describe the moves they make. A character without a weapon of some sort isn't going to trigger the hack and slash move when fighting a dragon since a bare-knuckle punch really doesn't do much to inch-thick scales. It doesn't count for the purposes of triggering the move.


 Likewise, sometimes equipment will avoid triggering a move. Climbing a sheer icy cliff is usually defying danger, but with a good set of climbing gear you might be able to avoid the imminent danger or calamity that triggers the move.


 Weapons are particularly likely to modify what moves you can trigger. A character with a dagger can easily stab the goblin gnawing on his leg, triggering hack and slash, but the character with a halberd is going to have a much harder time bringing it to bear on such a close foe.


 Items and gear of every sort have tags. Tags are terms to describe things. Some tags have a specific effect on the rules (things like damage reduction on armor or a magical bonus to a particular kind of move or stat). Other tags are purely about the fiction (like the close tag, which describes the length of a weapon and how near your enemies need to be for you to attack them). Tags help you describe your character's actions when the items are being used and they give the GM information about how the items you're using might go wrong or cause complications when you fail a roll.
\subsection{The Effects of Moves}


 The effects of moves are always about the fictional world the characters inhabit. A 10+ on hack and slash doesn't just mean the mechanical effects, it means you successfully attacked something and did some type of harm to it.


 Once you've figured out what the effects of the move are, apply them to the fiction and go back to the conversation. Always return to what's going on in the game.


 Some moves have immediate mechanical effects, like dealing damage or giving someone a bonus to their next roll. These effects are always a reflection of the fictional world the characters are in; make sure to use the fiction when describing the effects of the move.
\subsubsection{Some Moves \ldots }


  \ldots Use the phrase ``deal damage.'' Dealing damage means you roll the damage dice for your class; sometimes your weapon will add or subtract damage too. You use your damage dice any time you make an attack that could reasonably hurt your target. Usually that means you're wielding a weapon, but your fists can be weapons with the right training or an interesting situation


  \ldots Say ``take +1 forward.'' That means to take +1 to your next move roll (not damage). The bonus can be greater than +1, or even a penalty, like -1. There also might be a condition, such as ``take +1 forward to hack and slash,'' in which case the bonus applies only to the next time you roll hack and slash, not any other move.


  \ldots Say ``take +1 ongoing.'' That means to take +1 to all move rolls (not damage). The bonus can be larger than +1, or it can be a penalty, like -1. There also might be a condition, such as ``take +1 ongoing to volley.'' An ongoing bonus also says what causes it to end, like ``until you dismiss the spell'' or ``until you atone to your deity.''


  \ldots Give you ``hold.'' Hold is currency that allows you to make some choices later on by spending the hold as the move describes. Hold is always saved up for the move that generated it; you can't spend your hold from defend on trap expert or vice versa, for example.


  \ldots Present a choice. The choice you make, like all move effects, dictates things that happen in the fiction in addition to any more mechanical effects. The choice you make on the 10+ result of hack and slash to deal more damage at the cost of opening yourself up is exactly what's happening to your character: they have enough advantage that they can stay safe or push their luck.


  \ldots Give you a chance to say something about your character and their history. When you spout lore you may get asked how you know the information that the GM reveals. Take that opportunity to contribute to the game and show who your character really is. Just keep in mind the established facts and don't contradict anything that's already been described.


  \ldots Say ``mark XP\@.'' That means you add one to your current XP total.
\section*{Harm and Healing}


 Cuts, bruises, and mortal wounds are common dangers for adventurers to face in Dungeon World. In the course of play, characters will take damage, heal, and maybe even die. A character's health is measured by their hit points (HP). Damage subtracts from HP\@. In the right conditions, or with medical or magical help, damage is healed and HP is restored.
\subsection{HP}


 A character's HP is a measure of their stamina, endurance, and health. More HP means the character can fight longer and endure more trauma before facing Death's cold stare.


 Your class tells your maximum HP\@. Your Constitution (the ability, not the modifier) comes into play as well, so more Constitution means more HP\@. If your Constitution permanently changes during play you adjust your HP to reflect your new Constitution score. Unless your Constitution changes your maximum HP stays the same.
\subsection{Damage}
\index{Damage}


 When a character takes damage they subtract the damage dealt from their current HP\@. Armor mitigates damage; if a character has armor they subtract its value from the damage dealt. This might mean a blow is turned away completely--that's fine, it's what armor is for! Damage can never take a character below 0 HP\@.


 Damage is determined by the attacker. Player characters deal damage according to their class, the weapon used, and the move they've made.


 If a move just says ``deal damage'' the character rolls their class's damage dice plus any bonuses or penalties from moves, weapons, or effects. If a move specifies an amount of damage, use that in place of the class's damage roll.


 Monsters roll damage as listed in their description. Use this damage any time the monster takes direct action to hurt someone, even if they use a method other than their normal attack.


 Other sources of damage--like being struck by a chunk of a collapsing tower, or falling into a pit--are left to the GM based on these options:
\index{Damage!dice by severity}
\begin{itemize}
\item It threatens bruises and scrapes at worst: d4 damage
\item It's likely to spill some blood, but nothing horrendous: d6 damage
\item It might break some bones: d8 damage
\item It could kill a common person: d10 damage

\end{itemize}


 Add the \emph{ignores armor}
 tag if the source of the damage is particularly large or if the damage comes from magic or poison.


 Temporary or circumstantial armor works the same way as armor that you wear: 1 armor for partial cover, 2 armor for major cover.


 Damage is dealt based on the fiction. Moves that deal damage, like hack and slash, are just a special case of this: the move establishes that damage is being dealt in the fiction. Damage can be assigned even when no move is made, if it follows from the fiction.


 HP loss is often only part of the effect. If the harm is generalized, like falling into a pit, losing the HP is probably all there is to it. When the harm is specific, like an orc pulling your arm from its socket, HP should be part of the effect but not the entirety of it. The bigger issue is dealing with the newly busted arm: how do you swing a sword or cast a spell? Likewise having your head chopped off is not HP damage, it's just you being dead.
\subsubsection{Damage From Multiple Creatures}
\index{Damage!From Multiple Creatures}


 It's a brave monster that goes into battle alone. Most creatures fight with someone at their side, and maybe another at their back, and possibly an archer covering the rear, and so on. This can lead to multiple monsters dealing their damage at once.


 If multiple creatures attack at once roll the highest damage among them and add +1 damage for each monster beyond the first.

\begin{quote}
\emph{A goblin orkaster (d10+1 damage ignores armor) and three goblins (d6 damage) all throw their respective weapons--a magical acid orb for the orkaster, spears for the rest--at Lux as she assaults their barricade. I roll the highest damage, d10+1 ignores armor, and add +3 damage for the three other goblins. Adding it all up I tell Lux she takes 9 damage ignoring armor as the acid leaks into the scratches left by the spears.}
\end{quote} 
\subsubsection{Stun Damage}
\index{Damage!Stun}


 Stun damage is non-lethal damage. A PC who takes stun damage is defying danger to do anything at all, the danger being ``you're stunned.'' This lasts as long as makes sense in the fiction--you're stunned until you can get a chance to clear your head or fix whatever stunned you. A GM character that takes stun damage doesn't count it against their HP but will act accordingly, staggering around for a few seconds, fumbling blindly, etc.
\subsubsection{Adding and Subtracting Damage}
\index{Damage!Adding and Subtracting}


 When a move tells you to add damage, you add that damage to the roll on the dice. If it tells you to add some dice (like ``+1d4 damage'') you roll that extra dice and add its result to the total.


 The same goes for subtracting damage: you subtract the number from the total rolled. If you subtract a dice (like ``-1d6 damage'') you subtract the rolled amount from the original total. Damage never goes negative--0 damage is the minimum.
\subsubsection{Best and Worst}
\index{Damage!best (b[2dx])}
\index{Damage!worst (w[2dx])}

 Some monsters and moves have you roll damage multiple times and take the best or worst result. In this case roll as normal but only apply the best (or worst) result.


 If a monster rolls its d6 damage twice and takes the best result it's written b[2d6]. The b[] means ``best.'' Likewise, w[] means worst, so w[3d10] means ``roll a d10 for damage three times and use the worst result.''
\subsection{Healing}


 There are two sources of healing in Dungeon World: medical aid and the passage of time.


 Medical aid, both magical and mundane, heals damage according to the move or item used. Some moves may fully replenish HP while others heal just enough to keep someone standing through a fight.


 Whenever a character spends some time resting without doing anything to aggravate their wounds they heal. The amount of healing is described in the applicable moves: Make Camp for a night in a dangerous area, Recover for stays in civilization.


 No matter the source of the healing a character's HP can never increase above their maximum.
\subsection{Death}
\index{Death}

 Death stalks the edges of every battle. A character who is reduced to 0 HP immediately takes his Last Breath. Death comes for commoner and king alike--no stat is added to the Last Breath roll.


 No one knows what lies beyond the Black Gates of Death, but it is said that many secrets of the mortal plane are laid bare in the land of Death's dominion. When you die, you might just see them.


 Death offers bargains to some, from the simple to the costly. Death is capricious and may ask a favor in the future or exact a toll. He may demand a sacrifice or ask for something strange and seemingly innocent. Death's whim cannot be predicted.


 Depending on the outcome of the Last Breath the character may become stable. A stable character stays at 0 HP but is alive and unconscious. If they receive healing they regain consciousness and may return to battle or seek safety. If a stable character takes damage again they draw their Last Breath once more and return to face Death.
\subsubsection{After Death}


 Being an adventurer isn't easy--it's cold nights in the wild and sharp swords and monsters. Sooner or later, you're going to make that long walk to the Black Gates and give up the ghost. That doesn't mean you have to give it the satisfaction of sticking around. Death, in its way, is just another challenge to conquer. Even dead adventurers can rise again.


 If your character dies you can ask the GM and the other players to try and resurrect you. The GM will tell them what it will cost to return your poor, dead character to life. If you fulfill the GM's conditions the character is returned to life. The Resurrection spell is a special case of this: the magic of the spell gives you an easier way to get a companion back, but the GM still has a say.


 No matter the prospects of resurrection for now you make a new character. Maybe a hireling becomes a full-fledged adventurer worthy of a whole share and a part in the real action. Maybe the characters in the party find a new friend in a steading, willing to join them. Maybe your character had a vengeful family member who now seeks to take up their blades and spells to make right what happened. In any case, make your new character as you normally would at level 1. If your original character returns to life you can play either character, switching between them as you please (so long as it makes sense).


 GM, when you tell the players what needs to be done to bring their comrade back, don't feel like it has to derail the flow of the current game. Weave it in to what you know of the world. This is a great opportunity to change focus or introduce an element you've been waiting to show off. Don't feel, either, that it has to be some great and epic quest. If the character died at the end of a goblin pike, maybe all it takes is an awkward walk home and a few thousand gold pieces donated to a local temple. Think about the ramifications of such a charitable act and how it might affect the world. Remember: Death never forgets a soul stolen from his realm.
\subsection{Debilities}

\index{Damage!Debilities}

 Losing HP is a general thing, it's getting tired, bruised, cut, and so on. Some wounds are deeper though. These are debilities.


 \textbf{Weak (STR):}
 You can't exert much force. Maybe it's just fatigue and injury, or maybe your strength was drained by magic.


 \textbf{Shaky (DEX):}
 You're unsteady on your feet and you've got a shake in your hands.


 \textbf{Sick (CON):}
 Something just isn't right inside. Maybe you've got a disease or a wasting illness. Maybe you just drank too much ale last night and it's coming back to haunt you.


 \textbf{Stunned (INT):}
 That last knock to the head shook something loose. Brain not work so good.


 \textbf{Confused (WIS):}
 Ears ringing. Vision blurred. You're more than a little out of it.


 \textbf{Scarred (CHA):}
 It may not be permanent, but for now you don't look so good.


 Not every attack inflicts a debility--they're most often associated with magic, poison, or stranger things like a vampire sucking your blood. Each debility is tied to an ability and gives you -1 to that ability's modifier. The ability's score is unaffected so you don't have to worry about changing your maximum HP when you're sick.


 You can only have each debility once. If you're already Sick and something makes you Sick you just ignore it.


 Debilities are harder to heal than HP\@. Some high level magic can do it, sure, but your best bet is getting somewhere safe and spending a few days in a soft, warm bed. Of course, debilities are both descriptive and prescriptive: if something happens that would remove a debility, that debility is gone.


 Debilities don't replace d escriptions and using the established fiction. When someone loses an arm that doesn't mean they're Weak, it means they have one less arm. Don't let debilities limit you. A specific disease can have whatever effects you can dream up. Sick is just a convenient shorthand for some anonymous fever picked up from a filthy rat.
\section*{Magic}


 Dungeon World is a fantastic place: there's more to it than mud, blood, and ale in the tavern. Fire and wind conjured from the pure elements. Prayers for health, might, and divine retribution. ``Magic'' is the name given to those abilities not derived from the strength of man and beast but from forces beyond.


 Magic means many things. The druid's ability to take the shape of an animal is magic, as are the practiced effects of the wizard and the divine blessings of the cleric. Any ability that goes beyond the physically possible is magical.
\subsection{Spells}


 Some classes, like the cleric and the wizard have access to spells: specific magical effects that are the benefit of divine servitude or severe study. Each spell has a name, tags, a level, and an effect.


 The basic flow of magic is to know, prepare, cast, and forget a spell.


 Known spells are those a spellcaster has mastered enough to prepare. The cleric knows all cleric spells of their level or lower, including their rotes. The wizard starts knowing their cantrips and three 1st level spells. When the wizard gains a level they learn a new spell. The wizard stores their known spells in their spellbook.


 Even if a spellcaster knows a spell, they must have it prepared before they can cast it. With some time and concentration, as described in the Commune and Prepare moves, the spellcaster may choose known spells whose total levels are less than or equal to the caster's level plus 1 to prepare. The wizard always prepares their cantrips; the cleric always prepares their rotes. A prepared spell is ready to be cast.


 Casting a spell involves calling on a deity, chanting, waving ones hands, invoking mystical forces, and so on. To cast a spell you will usually make the cast a spell move. On a 10+ the spell takes effect, on a 7--9 the caster finds themself in trouble and must make a choice, but the spell is still cast. Some spells are ongoing--once they're cast they continue to have effect until something ends it.


 One option on a 7--9 result is to have the spell revoked or forgotten. A spell that is revoked or forgotten is still known, but no longer prepared, and therefore no longer castable. When the caster next Prepares or Communes they may choose the same spell again.
\section*{Character Change}


 Dungeon World is ever-changing. The characters change, too. As their adventures progress, player characters gain experience (XP), which lets them level up. This prepares them for greater danger, bigger adventures, and mightier deeds.


 Advancement, like everything else in Dungeon World, is both prescriptive and descriptive. Prescriptive means that when a player changes their character sheet the character changes in the fiction. Descriptive means that when the character changes in the fiction the player should change the character sheet to match.


 This isn't a benefit or detriment to the players or the GM; it's not an excuse to gain more powers or take them away. It's just a reflection of life in Dungeon World.

\begin{quote}
\emph{Avon, despite being a wizard, has risen to the notice of Lenoral, the deity of arcane knowledge. After being blessed by an avatar of Lenoral and saying his vows in the church, Avon is under the deity's watch. He can fulfill Petitions and gain boons like a cleric.}
\end{quote}

\begin{quote}
\emph{Gregor offers his signature weapon, an axe whose green steel is tempered in orc blood, as a desperate bargain to save King Authen from eternal damnation. Without his axe he gets none of the benefits of his signature weapon. Should he recover it he'll have access to its benefits again.}
\end{quote}


 Descriptive changes only happen when the character has clearly gained access to an ability. It's not up to any one player to decide this--if you think a character qualifies for a new ability, discuss it as a group.
\subsection{Level Up}


 As you play Dungeon World, you'll be doing three things most of all: exploring, fighting dangerous foes, and gathering treasure. For each of these things you'll be rewarded XP at the end of the session. Acting according to your alignment and fulfilling the conditions of your alignment moves will grant you XP at the end of each session as well. If you resolve a bond and create a new one, you'll gain XP, too. Any time you roll a 6- you get XP right away. The GM may have special conditions that you can fulfill to earn XP or might change the core ones to reflect the world. They'll let you know before you play.


 When your characters have safety and a chance to rest, they'll be able to make the Level Up move to level up and gain new moves.
\subsubsection{Multiclass Moves}


 The multiclass moves allow you to gain moves from another class. You get to choose any move of your level or lower. For the purpose of multiclassing, any starting class moves that depend on each other count as one move--the wizard's cast a spell, spellbook, and prepare spells for example. If a move from another class refers to your level, count your levels from the level where you first gained a move from that class.
\subsubsection{Requires and Replaces}


 Some moves that you gain at higher levels depend on other moves. If another move is listed along with the word Requires or Replaces you can only gain the new move if you have the listed move.


 A move that requires another move can only be taken if you have the move it requires already. You then have both moves and they both apply.


 A move that replaces another move can only be taken if you have the move it replaces already. You lose access to the replaced move and just have the new one. The new move will usually include all the benefits of the replaced one: maybe you replace a move that gives you 1 armor with one that gives you 2 armor instead.
\subsubsection{Beyond 10th Level}
\index{Characters Beyond 10th Level}

 Once you've reached 10th level things change a little. When you have enough XP to go to 11th level instead you choose one of these:
\begin{itemize}
\item Retire to safety
\item Take on an apprentice
\item Change entirely to a new class

\end{itemize}


 If you retire you create a new character to play instead and work with the GM to establish your place in the world. If you take on an apprentice you play a new character (the apprentice) alongside your current character, who stops gaining XP\@. Changing classes means keeping your ability scores, race, HP, and whatever moves you and the GM agree are core to who your character is. You lose all other class moves, replacing them with the starting moves of your new class.
\section*{Bonds}
\index{Bonds}

 Bonds are what make you a party of adventurers, not just a random assortment of people. They're the feelings, thoughts, and shared history that tie you together. You will always have at least one bond, and you'll often have more.


 Each bond is a simple statement that relates your character to another player character. Your class gives you a few to start with, you'll replace your starting bonds and gain new ones through play.
\subsection{Resolving Bonds}
\index{Bonds!Resolving}

 At the end of each session you may resolve one bond. Resolution of a bond depends on both you and the player of the character you share the bond with: you suggest that the bond has been resolved and, if they agree, it is. When you resolve a bond, you get to mark XP\@.


 A bond is resolved when it no longer describes how you relate to that person. That may be because circumstances have changed--Thelian used to have your back but after he abandoned you to the goblins, you're not so sure. Or it could be because that's no longer a question--you guided Wesley before and he owed you, but he paid that debt when he saved your life with a well-timed spell. Any time you look at a bond and think ``that's not a big factor in how we relate anymore'' the bond is at a good place to resolve.


 If you have a blank bond left over from character creation you can assign a name to it or write a new bond in its place whenever you like. You don't get an XP for doing so, but you do get more defined bonds to resolve in the future.
\subsection{Writing New Bonds}
\index{Bonds!Writing}


 You write a new bond whenever you resolve an old one. Your new bond may be with the same character, but it doesn't have to be.


 When you write a new bond choose another character. Pick something relevant to the last session--maybe a place you traveled together or a treasure you discovered. Choose a thought or belief your character holds that ties the two together and an action, something you're going to do about it. You'll end up with something like this:


\begin{quote}
\emph{Mouse's quick thinking saved me from the white dragon we faced. I owe her a boon.}
\end{quote}


\begin{quote}
\emph{Avon proved himself a coward in the dungeons of Xax'takar. He is a dangerous liability to the party and must be watched.}
\end{quote}


\begin{quote}
\emph{Valeria's kindness to the Gnomes of the Vale has swayed my heart, I will prove to her I am not the callous fiend she thinks I am.}
\end{quote}

\begin{quote}
\emph{Xotoq won the Bone-and-Whispers Axe through trickery! It will be mine, I swear it.}
\end{quote}


 These new bonds are ju st like the old ones--use them, resolve them, replace them.
\section*{Alignment}
\index{Alignment}

 Alignment is your character's way of thinking and moral compass. For the character, this can be an ethical ideal, religious strictures, or maybe just a gut instinct. It reflects the things your character might aspire to be and can guide you when you're not sure what to do next. Some characters might proudly proclaim their alignment while others might hide it away. A character might not say, ``I'm an evil person,'' but may instead say, ``I put myself first.'' That's all well and good for a character, but the world knows otherwise. Buried deep down inside is the ideal self a person wants to become--it is this mystic core that certain spells and abilities tap into when detecting someone's alignment. Every sentient creature in Dungeon World bears an alignment, be they an elf, a human, or some other, stranger thing.


 The alignments are Good, Lawful, Neutral, Chaotic, and Evil. Each one shows an aspiration to be a different type of person.


 Lawful creatures aspire to impose order on the world, either for their own benefit or for that of others. Chaotic creatures embrace change and idealize the messy reality of the world, prizing freedom above all else. Good creatures seek to put others before themselves. Evil creatures put themselves first at the expense of others.


 A Neutral creature looks out for itself so long as that doesn't jeopardize someone else's well-being. Neutral characters are content to live their lives and pursue their own goals and let others do the same.


 Most creatures are Neutral. They take no particular pleasure in harming others, but will do it if it is justified by their situation. Those that put an ideal--be it Law, Chaos, Good, or Evil--above themselves are harder to find.


 Even two creatures of the same alignment can come into conflict. Aspiring to help others does not grant infallibility, two Good creatures may fight and die over two different views of how to do right.
\subsection{Changing Alignment}
\index{Alignment!Changing}

 Alignment can, and will, change. Usually such a change comes about as a gradual move toward a decisive moment. Any time a character's view of the world has fundamentally shifted they can chose a new alignment. The player must have a reason for the change which they can explain to the other players.


 In some cases a player character may switch alignment moves while still keeping the same alignment. This reflects a smaller shift, one of priority instead of a wholesale shift in thinking. They simply choose a new move for the same alignment from below and mention why their character now sees this as important.
\index{Alignment!Moves}
\subsubsection{Lawful}
\begin{itemize}
\item Uphold the letter of the law over the spirit
\item Fulfill a promise of import
\item Bring someone to justice
\item Choose honor over personal gain
\item Return treasure to its rightful owner

\end{itemize}
\subsubsection{Good}
\begin{itemize}
\item Ignore danger to aid another
\item Lead others into righteous battle
\item Give up powers or riches for the greater good
\item Reveal a dangerous lie
\item Show mercy

\end{itemize}
\subsubsection{Neutral}
\begin{itemize}
\item Make an ally of someone powerful
\item Defeat a personally important foe
\item Learn a secret about an enemy
\item Uncover a hidden truth

\end{itemize}
\subsubsection{Chaotic}
\begin{itemize}
\item Reveal corruption
\item Break an unjust law to benefit another
\item Defeat a tyrant
\item Reveal hypocrisy

\end{itemize}
\subsubsection{Evil}
\begin{itemize}
\item Take advantage of someone's trust
\item Cause suffering for its own sake
\item Destroy something beautiful
\item Upset the rightful order
\item Harm an innocent

\end{itemize} 
\section*{Hirelings}


 Hirelings are those sorry souls that--for money, glory, or stranger needs--venture along with adventurers into the gloom and danger. They are the foolhardy that seek to make their name as adventurers.


 Hirelings serve a few purposes. To the characters, they're the help. They lend their strength to the player characters' efforts in return for their pay. To the players, they're a resource. They buy the characters some extra time against even the most frightening of threats. They're also replacement characters, waiting to step up into the hero's role when a player character falls. To the GM, they're a human face for the characters to turn to, even in the depths of the earth or the far reaches of the planes.


 Hirelings are not heroes. A hireling may become a hero, as a replacement character, but until that time they're just another GM character. As such their exact HP, armor, and damage aren't particularly important. A hireling is defined by their \textbf{Skill}
 (or Skills) a \textbf{Cost}
 and a \textbf{Loyalty}
 score.


 A hireling's skill is a special benefit they provide to the players. Most skills are related to class abilities, allowing a hireling to fill in for a certain class. If you don't have a ranger but you need to track the assassin's route out of Torsea anyway, you need a Tracker. Each skill has a rank, usually from 1 to 10. The higher the rank the more trained the hireling. Generally hirelings only work for adventurers of equal or higher level than their highest skill.


 Skills don't limit what a hireling can do, they just provide mechanics for a certain ability. A hireling with the protector skill can still carry your burdens or check for traps, but the outcome isn't guaranteed by a rule. It will fall entirely to the circumstances and the GM\@. Sending a hireling to do something that is clearly beyond their abilities is asking the GM for trouble.


 No hireling works for free. The hireling's cost is what it takes to keep them with the player characters. If the hireling's cost isn't paid regularly (usually once a session) they're liable to quit or turn on their employers.


 When hirelings are in play, the players may have to make the Order Hirelings move. The move uses the loyalty of the hireling that triggered the move:

\HRule
 Hirelings do what you tell them to, so long as it isn't obviously dangerous, degrading, or stupid, and their cost is met. \textbf{When a hireling find themselves in a dangerous, degrading, or just flat-out crazy situation due to your orders}
 roll+loyalty. On a 10+ they stand firm and carry out the order. On a 7--9 they do it for now, but come back with serious demands later. Meet them or the hireling quits on the worst terms.
\HRule
\section*{Making a Hireling}


 Hirelings are easy to make on the fly. When someone enters the players' employ note down their name and what cost they've agreed to as well as any skills they may have.


 Start with a number based on where the hireling was found. Hirelings in villages start with 2--5. Town hirelings get 4--6. Keep hirelings are 5--8. City hirelings are 6--10. Distribute the hireling's number between loyalty, a main skill, and zero or more secondary skills. Starting loyalty higher than 2 is unusual, as is starting loyalty below 0. Choose a cost for the hireling and you're done.


 A hireling's stats, especially their loyalty, may change during play as a reflection of events. A particular kindness or bonus from the players is worth +1 loyalty forward. Disrespect is -1 loyalty forward. If it's been a while since their cost was last paid they get -1 loyalty ongoing until their cost is met. A hireling's loyalty may be permanently increased when they achieve some great deed with the players. A significant failure or beating may permanently lower the hireling's loyalty.
\subsection{Costs}
\begin{itemize}
\item The Thrill of Victory
\item Money
\item Uncovered Knowledge
\item Fame and Glory
\item Debauchery
\item Good Accomplished

\end{itemize}
\subsection{Skills}


 When you make a hireling, distribute points among one or more of these skills.
\subsubsection{Adept}
\index{Adept (hireling skill)}

 An adept has at least apprenticed to an arcane expert, but is not powerful in their own right. They're the grad students of the arcane world.


 \emph{Arcane Assistance}
--When an adept aids in the casting of a spell of lower level than their skill, the spell's effects have greater range, duration, or potency. The exact effects depend on the situation and the spell and are up to the GM\@. The GM will describe what effects the assist will add before the spell is cast. The most important feature of casting with an adept is that any negative effects of the casting are focused on the adept first.
\subsubsection{Burglar}
\index{Burglar (hireling skill)}

 Burglars are skilled in a variety of areas, most of them illicit or dangerous. They are good with devices and traps, but not too helpful in the field of battle.


 \emph{Experimental Trap Disarming}
--When a burglar leads the way they can detect traps almost in time. If a trap would be sprung while a burglar is leading the way the burglar suffers the full effects but the players get +skill against the trap and add the burglar's skill to their armor against the trap. Most traps leave a burglar in need of immediate healing. If the players Make Camp near the trap, the burglar can disarm it by the time camp is broken.
\subsubsection{Minstrel}
\index{Minstrel (hireling skill)}

 When a smiling face is needed to smooth things over or negotiate a deal a minstrel is always happy to lend their services for the proper price.


 \emph{A Hero's Welcome}
--When you enter a place of food, drink, or entertainment with a minstrel you will be treated as a friend by everyone present (unless your actions prove otherwise). You also subtract the minstrel's skill from all prices in town.
\subsubsection{Priest}
\index{Priest (hireling skill)}

 Priests are the lower ranking clergy of a religion, performing minor offices and regular sacraments. While not granted spells themselves, they are able to call upon their deity for minor aid.


 \emph{Ministry}
--When you make camp with a priest if you would normally heal you heal +skill HP\@.


 \emph{First Aid}
--When a priest staunches your wounds heal 2\~A�skill HP\@. You take -1 forward as their healing is painful and distracting.
\subsubsection{Protector}
\index{Protector (hireling skill)}


 A protector stands between their employer and the blades, fangs, teeth, and spells that would harm them. 


 \emph{Sentry}
--When a protector stands between you and an attack you increase your armor against that attack by the defender's skill, then reduce their skill by 1 until they receive healing or have time to mend.


 \emph{Intervene}
--When a protector helps you defy danger you may opt to take +1 from their aid. If you do you cannot get a 10+ result, a 10+ instead counts as a 7--9.
\subsubsection{Tracker}
\index{Tracker (hireling skill)}


 Trackers know the secrets of following a trail, but they don't have the experience with strange creatures and exotic locales that make for a great hunter.


 \emph{Track}
--When a tracker is given time to study a trail while Making Camp, when camp is broken they can follow the trail to the next major change in terrain, travel, or weather.


 \emph{Guide}
--When a tracker leads the way you automatically succeed on any Perilous Journey of a distance (in rations) lower than the tracker's skill.
\subsubsection{Warrior}
\index{Warrior (hireling skill)}


 Warriors are not masters of combat, but they are handy with a weapon.


 \emph{Man-at-arms}
--When you deal damage while a warrior aids you add their skill to the damage done. If your attack results in consequences (like a counter attack) the man-at-arms takes the brunt of it.
\section*{The Adventurer's Life}


 Now you know the basics. It's time you found out what the adventurer's life is really like. They say it's all gold and glory. That's sometimes true, but sometimes it also means digging through otyugh waste for a chance at one more gold coin.
\subsection{Dungeons}
\index{Dungeons}

 As an adventurer you'll spend a lot of time in dungeons. The word ``dungeon'' conjures up an image of the stony halls under a castle where prisoners are kept, but a dungeon is really any place filled with danger and opportunity. A dragon's cave, an enemy camp, a forgotten sewer, a sky castle, the very foundations of the world.


 The most important thing to remember when you're in a dungeon is that it's a living place. Just because you cleared the guards out of the entryway doesn't mean they won't be replaced by fresh recruits. Every monster, soldier, or leader you kill has friends, mates, followers, and spawn somewhere. Don't count on anything in a dungeon.


 Since dungeons are living places you'd better prepare for the long haul. Rations are your best friend. Delving into the Hall of Xa'th'al isn't a day trip. Once you're inside your exit might be blocked. Even if you could just waltz out the time you spend doing it just gives your enemies time to prepare. Those goblins aren't tough, but when they have time to rally and prepare traps \ldots 


 Speaking of traps--keep your eyes open for them, too. The thief is your best friend there. They can stop you before you wander into a pit trap or fill the room with acid. Without one you're not in dire trouble, but you're likely to need to take your time and be extra careful. You can investigate an area by discerning realities, but you'll be taking more risks than a skilled thief would.


 When you're unlucky enough to trigger a trap you might have a chance to get out of the way, throw up a quick protective spell, or save a friend--most likely by defying danger. Of course not every trap is so crude as to give you time to get out of the way. A well-built trap will have a blade in your side before you even know it's sprung.


 That sounds grim, sure, but it's not as bad as all that. You've got steel, skills, and spells. If you stick together and keep your wits you'll make it out alive. Probably.
\subsection{Monsters}


 The beasts and worse that fill dungeons? We call them monsters.


 Not all of them appear monstrous. Sometimes it's just a guy in some armor--no horns, flames, or wings, nothing. But when that guy wants to kill you, well, he's as much a monster as the rest.


 Some don't even need arms and armor. A wily warlock or nefarious noble can stab you in the back a dozen times with a word or two. Be wary of anyone who can stroll around a dungeon with nothing but a robe and a staff: there's a reason they don't need a shell of steel.


 When it comes to fighting monsters, it's an even bet: your life versus theirs. You should know that going into it. If you can avoid it, never fight with even odds. Unless you have the advantage you're probably better off working to gain that advantage than betting your life on a fight. Find their weaknesses, pad your advantages, and you'll live long enough to enjoy the spoils.


 Fights often mean triggering moves like hack and slash, defend, or volley. Defy danger comes up pretty often too, and class moves like cast a spell. The best fight for you is one where you have the drop--since hack and slash is triggered by attacking in melee, and a defenseless enemy isn't really in melee, the move won't trigger--you'll just bury a weapon or spell in their back and deal your damage.


 Monsters generally fall into a few types. Humanoids are more or less like you--orcs, goblins, and so on. Beasts are animals, but not so docile as Bessie the cow: think foot-long horns and acid sacs. Constructs are crafted life. Planar monsters come from beyond this world, from places only dreamed of. The undead might be the worst of all: that which is dead is damn hard to kill again.


 When you find yourself in a fight with a monster you have a few different tricks up your sleeve that can help you survive. If the monster's something you might know about, you could consult your knowledge and spout lore. It never hurts to take a minute to look around and discern realities, too--there might be something helpful nearby that you missed. Make sure you understand your class moves and how they can help you, too. You never know when a move might come in handy in a new way.
\subsection{Wilderness}


 There's dungeons, there's civilization, and there's all the stuff in-between: the wilderness. The line between a forest and a dungeon is thinner than you might think--have you ever been lost in the night and surrounded by wolves?


 Journeys by road are easy. When you've got a trail to follow and some modicum of protection you're not even making moves--you just consume some rations on the way and make it to your destination. If it's a perilous journey though \ldots 


 On a perilous journey you'll need a trailblazer, a scout, and a quartermaster. That means you'll probably want at least three people when you're traveling in dangerous areas. Fewer than three and you'll be neglecting something--that's an invitation for trouble.
\subsection{Friends and Enemies}


 You're an adventurer, so people will pay attention to you. Not all of that attention is going to be positive. You'll find that, especially once you're laden down with ancient treasure, all manner of hangers-on will appear from the woodwork.


 Sure, you can get leverage on these people and parley them to get what you want, but the way to build a lasting connection is to do right by them. Forcing Duke Alhoro to give you a castle in return for his daughter will get you the land, but the reputation that comes along with your shady dealing won't do you many favors. Coercion isn't mind control, so play it nice if you want to make friends.


 Magic, though, that just might be mind control. The morality of it's debatable but you can bend someone to your will if you don't mind tossing their free will in the corner.


 It's worth keeping track of who's got your back and who'd sooner stab you in it. The GM will be doing the same, and the worst enemy is the one you don't know. You're not the only ones in Dungeon World with grand designs.


 While you live the adventurer's life, with no fixed address to give, other folks are likely to be more settled. Knowing where the blacksmith is that does the best work, or which town's inns will put you up free of charge, is a fine thing indeed.


 Keep in mind that not all power is physical. Even if you could take down King Arlon in a fight you'll just be inviting retribution from his kin, allies, and court. Station is its own kind of power apart from magic and might.
\subsection{The World}


 You're an adventurer; you're a big deal. But there are other forces at work too. The world will go on without you. If you don't deal with the goblin infestation in the sewers maybe someone else will. Or maybe the goblins will take over the city. Do you really want to find out?


 A world in motion is a world waiting to be changed. Your choices of who to kill (or not), where to go, what bargains to make--it all changes the world you're in. Changing the world requires acting on it--making moves and pursuing treasure and exploration. Change comes in many forms, including XP used to level up and gain new abilities. Those abilities are then used to go back out into the world and stir things up. It's a cycle of change and growth for both you and the world you live in.

 
