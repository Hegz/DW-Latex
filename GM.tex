\chapter{The GM}

There are many different fantasy genres, each with their own style or advice for GMing. Dungeon World is designed for one of those styles in particular--a world of elves, orcs, dragons and magic where dark dangers mix with lighthearted adventure. The rules in this chapter will help you run a game in that style.

The characters have rules to follow when they roll dice and take actions. The GM has rules to follow, too. You'll be refereeing, adjudicating, and describing the world as you go--Dungeon World provides a framework to guide you in doing so.

This chapter isn't about advice for the GM or optional tips and tricks on how best to play Dungeon World. It's a chapter with procedures and rules for whoever takes on the role of GM.
\subsection{GMing Dungeon World: A Framework}

Running a game of Dungeon World is built on these: the GM's \textbf{agenda}, \textbf{principles}, and \textbf{moves}. The agenda is what you set out to do when you sit down at the table. The principles are the guides that keep you focused on that agenda. The GM's moves are the concrete, moment-to-moment things you do to move the game forward. You'll make moves when players miss their rolls, when the rules call for it, and whenever the players look to you to see what happens. Your moves keep the fiction consistent and the game's action moving forward.

The GM's agenda, principles, and moves are rules just like damage or stats or HP\@. You should take the same care in altering them or ignoring them that you would with any other rule.
\section*{How to GM}

When you sit down at the table as a GM you do these things:
\begin{itemize}
\item Describe the situation
\item Follow the rules
\item Make moves
\item Exploit your prep

\end{itemize}

The players have it easy--they just say what their characters say, think, and do. You have it a bit harder. You have to say everything else. What does that entail?

First and foremost, you \textbf{describe the immediate situation around the players at all times}. This is how you start a session, how you get things rolling after a snack break, get back on track after a great joke: tell them what the situation is in concrete terms.

Use detail and senses to draw them in. The situation isn't just an orc charging you, it's an orc painted in blood swinging a hammer and yelling bloody murder. You can leverage a lack of information, too. The sound of clattering armor and shuffling feet, for instance.

The situation around them is rarely ``everything's great, nothing to worry about.'' They're adventurers going on adventures--give them something to react to.

When you describe the situation, always end with ``What do you do?'' Dungeon World is about action and adventure! Portray a situation that demands a response.

From the get-go make sure to \textbf{follow the rules}. This means your GM rules, sure, but also keep an eye on the players' moves. It's everyone's responsibility to watch for when a move has been triggered, including you. Stop the players and ask if they mean to trigger the rules when it sounds like that's what they're doing.

Part of following the rules is \textbf{making moves}. Your moves are different than player moves and we'll describe them in detail in a bit. Your moves are specific things you can do to change the flow of the game.

In all of these things, \textbf{exploit your prep}. At times you'll know something the players don't yet know. You can use that knowledge to help you make moves. Maybe the wizard tries to cast a spell and draws unwanted attention. They don't know that the attention that just fell on them was the ominous gaze of a demon waiting two levels below, but you do. 
\section*{Agenda}
\index{GM!Agenda}

Your agenda makes up the things you aim to do at all times while GMing a game of Dungeon World:
\begin{itemize}
\item Portray a fantastic world
\item Fill the characters' lives with adventure
\item Play to find out what happens

\end{itemize}

Everything you say and do at the table (and away from the table, too) exists to accomplish these three goals and no others. Things that aren't on this list aren't your goals. You're not trying to beat the players or test their ability to solve complex traps. You're not here to give the players a chance to explore your finely crafted setting. You're not trying to kill the players (though monsters might be). You're most certainly not here to tell everyone a planned-out story.

Your first agenda is to \textbf{portray a fantastic world}. Dungeon World is all about guts, guile, and bravery against darkness and doom. It's about characters who have decided to take up a life of adventure in the hopes of some glorious reward. It's your job to participate in that by showing the players a world in which their characters can find that adventure. Without the player characters the world would fall into chaos or destruction--it might still even with them. It's up to you to portray the fantastic elements of that world. Show the players the wonders of the world they're in and encourage them to react to it.

\textbf{Filling the characters' lives with adventure}
means working with the players to create a world that's engaging and dynamic. Adventurers are always caught up in some world-threatening danger or another--encourage and foster that kind of action in the game.

Dungeon World adventures \textbf{never}
presume player actions. A Dungeon World adventure portrays a setting in motion--someplace significant with creatures big and small pursuing their own goals. As the players come into conflict with that setting and its denizens, action is inevitable. You'll honestly portray the repercussions of that action.

This is how you \textbf{play to find out what happens}. You're sharing in the fun of finding out how the characters react to and change the world you're portraying. You're all participants in a great adventure that's unfolding. So really, don't plan too hard. The rules of the game will fight you. It's fun to see how things unfold, trust us.


\section*{Principles}
\index{GM!principles}
\begin{itemize}
\item Draw maps, leave blanks
\item Address the characters, not the players
\item Embrace the fantastic
\item Make a move that follows
\item Never speak the name of your move
\item Give every monster life
\item Name every person
\item Ask questions and use the answers
\item Be a fan of the characters
\item Think dangerous
\item Begin and end with the fiction
\item Think offscreen, too

\end{itemize}

Your principles are your guides. Often, when it's time to make a move, you'll already have an idea of what makes sense. Consider it in light of your principles and go with it, if it fits.
\subsubsection{Draw maps, leave blanks}

Dungeon World exists mostly in the imaginations of the people playing it; maps help everyone stay on the same page. You won't always be drawing them yourself, but any time there's a new location described make sure it gets added to a map.

When you draw a map don't try to make it complete. Leave room for the unknown. As you play you'll get more ideas and the players will give you inspiration to work with. Let the maps expand and change.
\subsubsection{Address the characters, not the players}

Addressing the characters, not the players, means that you don't say, ``Tony, is Dunwick doing something about that wight?'' Instead you say, ``Dunwick, what are you doing about the wight?'' Speaking this way keeps the game focused on the fiction and not on the table. It's important to the flow of the game, too. If you talk to the players you may leave out details that are important to what moves the characters make. Since moves are always based on the actions of the character you need to think about what's happening in terms of those characters--not the players portraying them.
\subsubsection{Embrace the fantastic}

Magic, strange vistas, gods, demons, and abominations: the world is full of mystery and magic. Embrace that in your prep and in play. Think about ``the fantastic'' on various scales. Think about floating cities or islands crafted from the corpse of a god. Think about village wise-men and their spirit familiars or the statue that the local bandits touch to give them luck. The characters are interesting people, empowered by their gods, their skill at arms, or by mystical training. The world should be just as engaging.
\subsubsection{Make a move that follows}

When you make a move what you're actually doing is taking an element of the fiction and bringing it to bear against the characters. Your move should always follow from the fiction. They help you focus on one aspect of the current situation and do something interesting with it. What's going on? What move makes sense here?
\subsubsection{Never speak the name of your move}

There is no quicker way to ruin the consistency of Dungeon World than to tell the players what move you're making. Your moves are prompts to you, not things you say directly.

You never show the players that you're picking a move from a list. You know the reason the slavers dragged off Omar was because you made the ``put someone in a spot'' move, but you show it to the players as a straightforward outcome of their actions, since it is.
\subsubsection{Give every monster life}

Monsters are fantastic creatures with their own motivations (simple or complex). Give each monster details that bring it to life: smells, sights, sounds. Give each one enough to make it real, but don't cry when it gets beat up or overthrown. That's what player characters do!
\subsubsection{Name every person}

Anyone that the players speak with has a name. They probably have a personality and some goals or opinions too, but you can figure that out as you go. Start with a name. The rest can flow from there.
\subsubsection{Ask questions and use the answers}

Part of playing to find out what happens is explicitly not knowing everything, and being curious. If you don't know something, or you don't have an idea, ask the players and use what they say.

The easiest question to use is ``What do you do?'' Whenever you make a move, end with ``What do you do?'' You don't even have to ask the person you made the move against. Take that chance to shift the focus elsewhere: ``Rath's spell is torn apart with a flick of the mage's wand. Finnegan, that spell was aiding you. What are you doing now that it's gone?''
\subsubsection{Be a fan of the characters}

Think of the players' characters as protagonists in a story you might see on TV\@. Cheer for their victories and lament their defeats. You're not here to push them in any particular direction, merely to participate in fiction that features them and their action.
\subsubsection{Think dangerous}

Everything in the world is a target. You're thinking like an evil overlord: no single life is worth anything and there is nothing sacrosanct. Everything can be put in danger, everything can be destroyed. Nothing you create is ever protected. Whenever your eye falls on something you've created, think how it can be put in danger, fall apart or crumble. The world changes. Without the characters' intervention, it changes for the worse.
\subsubsection{Begin and end with the fiction}

Everything you and the players do in Dungeon World comes from and leads to fictional events. When the players make a move, they take a fictional action to trigger it, apply the rules, and get a fictional effect. When you make a move it always comes from the fiction.
\subsubsection{Think offscreen too}

Just because you're a fan of the characters doesn't mean everything happens right in front of them. Sometimes your best move is in the next room, or another part of the dungeon, or even back in town. Make your move elsewhere and show its effects when they come into the spotlight.
\section*{Moves}
\index{GM!moves}
\index{Moves!GM|(}

Whenever everyone looks to you to see what happens choose one of these. Each move is something that occurs in the fiction of the game--they aren't code words or special terms. ``Use up their resources'' literally means to expend the resources of the characters, for example.
\begin{itemize}
\item Use a monster, danger, or location move
\item Reveal an unwelcome truth
\item Show signs of an approaching threat
\item Deal damage
\item Use up their resources
\item Turn their move back on them
\item Separate them
\item Give an opportunity that fits a class' abilities
\item Show a downside to their class, race, or equipment
\item Offer an opportunity, with or without cost
\item Put someone in a spot
\item Tell them the requirements or consequences and ask

\end{itemize}

Never speak the name of your move (that's one of your principles). Make it a real thing that happens to them: ``As you dodge the hulking ogre's club, you slip and land hard. Your sword goes sliding away into the darkness. You think you saw where it went but the ogre is lumbering your way. What do you do?''

No matter what move you make, always follow up with ``What do you do?'' Your moves are a way of fulfilling your agenda--part of which is to fill the characters' lives with adventure. When a spell goes wild or the floor drops out from under them adventurers react or suffer the consequences of inaction.
\subsection{When to Make a Move}

You make a move:
\begin{itemize}
\item When everyone looks to you to find out what happens
\item When the players give you a golden opportunity
\item When they roll a 6-

\end{itemize}

Generally when the players are just looking at you to find out what happens you make a soft move, otherwise you make a hard move.

A soft move is one without immediate, irrevocable consequences. That usually means it's something not all that bad, like revealing that there's more treasure if they can just find a way past the golem (offer an opportunity with cost). It can also mean that it's something bad, but they have time to avoid it, like having the goblin archers loose their arrows (show signs of an approaching threat) with a chance for them to dodge out of danger.

A soft move ignored becomes a golden opportunity for a hard move. If the players do nothing about the hail of arrows flying towards them it's a golden opportunity to use the deal damage move.

Hard moves, on the other hand, have immediate consequences. Dealing damage is almost always a hard move, since it means a loss of HP that won't be recovered without some action from the players.

When you have a chance to make a hard move you can opt for a soft one instead if it better fits the situation. Sometimes things just work out for the best.
\subsection{Choosing a Move}

To choose a move, start by looking at the obvious consequences of the action that triggered it. If you already have an idea, think on it for a second to make sure it fits your agenda and principles and then do it. \textbf{Let your moves snowball}. Build on the success or failure of the characters' moves and on your own previous moves.

If your first instinct is that this won't hurt them now, but it'll come back to bite them later, great! That's part of your principles (think offscreen too). Make a note of and reveal it when the time is right.
\subsection{Making your Move}

When making a move, keep your principles in mind. In particular, never speak the name of your move and address the characters, not the players. Your moves are not mechanical actions happening around the table. They are concrete events happening to the characters in the fictional world you are describing.

Note that ``deal damage'' is a move, but other moves may include damage as well. When an ogre flings you against a wall you take damage as surely as if he had smashed you with his fists.

After every move you make, always ask ``What do you do?''
\subsubsection{Use a monster, danger, or location move}

Every monster in an adventure has moves associated with it, as do many locations. A monster or location move is just a description of what that location or monster does, maybe ``hurl someone away'' or ``bridge the planes.'' If a player move (like hack and slash) says that a monster gets to make an attack, make an aggressive move with that monster.

The overarching dangers of the adventure also have moves associated with them. Use these moves to bring that danger into play, which may mean more monsters.
\subsubsection{Reveal an unwelcome truth}

An unwelcome truth is a fact the players wish wasn't true: that the room's been trapped, maybe, or that the helpful goblin is actually a spy. Reveal to the players just how much trouble they're in.
\subsubsection{Show signs of an approaching threat}

This is one of your most versatile moves. ``Threat'' means anything bad that's on the way. With this move, you just show them that something's going to happen unless they do something about it.
\subsubsection{Deal damage}

When you deal damage, choose one source of damage that's fictionally threatening a character and apply it. In combat with a lizard man? It stabs you. Triggered a trap? Rocks fall on you.

The amount of damage is decided by the source. In some cases, this move might involve trading damage both ways, with the character also dealing damage.

Most damage is based on a die roll. When a player takes damage, tell them what to roll. You never need to touch the dice. If the player is too cowardly to find out their own fate, they can ask another player to roll for them.
\subsubsection{Use up their resources}

Surviving in a dungeon, or anywhere dangerous, often comes down to supplies. With this move, something happens to use up some resource: weapons, armor, healing, ongoing spells. You don't always have to use it up permanently. A sword might just be flung to the other side of the room, not shattered.
\subsubsection{Turn Their Move Back On Them}

Think about the benefits a move might grant a character and turn them around in a negative way. Alternately, grant the same advantage to someone who has it out for the characters. If Ivy has learned of Duke Horst's men approaching from the east, maybe a scout has spotted her, too.
\subsubsection{Separate Them}

There are few things worse than being in the middle of a raging battle with blood-thirsty owlbears on all sides--one of those things is being in the middle of that battle with no one at your back.

Separating the characters can mean anything from being pushed apart in the heat of battle to being teleported to the far end of the dungeon. Whatever way it occurs, it's bound to cause problems.
\subsubsection{Give an opportunity that fits a class' abilities}

The thief disables traps, sneaks, and picks locks. The cleric deals with the divine and the dead. Every class has things that they shine at--present an opportunity that plays to what one class shines at.

It doesn't have to be a class that's in play right now though. Sometimes a locked door stands between you and treasure and there's no thief in sight. This is an invitation for invention, bargaining, and creativity. If all you've got is a bloody axe doesn't every problem look like a skull?
\subsubsection{Show a downside to their class, race, or equipment}

Just as every class shines, they all have their weaknesses too. Do orcs have a special thirst for elven blood? Is the cleric's magic disturbing dangerous forces? The torch that lights the way also draws attention from eyes in the dark.
\subsubsection{Offer an opportunity, with or without cost}

Show them something they want: riches, power, glory. If you want, you can associate some cost with it too, of course.

Remember to lead with the fiction. You don't say, ``This area isn't dangerous so you can make camp here, if you're willing to take the time.'' You make it a solid fictional thing and say, ``Helferth's blessings still hang around the shattered altar. It's a nice safe spot, but the chanting from the ritual chamber is getting louder. What do you do?''
\subsubsection{Put someone in a spot}

A spot is someplace where a character needs to make tough choices. Put them, or something they care about, in the path of destruction. The harder the choice, the tougher the spot.
\subsubsection{Tell them the requirements or consequences and ask}

This move is particularly good when they want something that's not covered by a move, or they've failed a move. They can do it, sure, but they'll have to pay the price. Or, they can do it, but there will be consequences. Maybe they can swim through the shark-infested moat before being devoured, but they'll need a distraction. Of course, this is made clear to the characters, not just the players: the sharks are in a starved frenzy, for example.
\index{Moves!GM|)}
\section*{Dungeon Moves}
\index{Moves!dungeon (GM)|(}

Dungeon Moves are a special subset that are used to make or alter a dungeon on the fly. Use these if your players are exploring a hostile area that you don't already have planned completely.

Map out the area being explored as you make these moves. Most of them will require you to add a new room or element to your map.
\begin{itemize}
\item Change the environment
\item Point to a looming threat
\item Introduce a new faction or type of creature
\item Use a threat from an existing faction or type of creature
\item Make them backtrack
\item Present riches at a price
\item Present a challenge to one of the characters

\end{itemize}

You can make these moves whenever everyone looks to you to say something, when the players present you an opportunity, or when the players miss on a roll. They're particularly well-suited for when the characters enter a new room or hallway and want to know what they find there.
\subsubsection{Change the environment}

The environment is the general feel of the area the players are in: carved tunnels, warped trees, safe trails, or whatever else. This is your opportunity to introduce them to a new environment: the tunnels gradually become naturally carved, the trees are dead and strange, or the trails are lost and the wilderness takes over. Use this move to vary the types of areas and creatures the players will face.
\subsubsection{Point to a looming threat}

If you know that something is lurking and waiting for the players to stumble upon it, this move shows them the signs and clues. This move is the dragon's footprints in the mud or the slimy trail of the gelatinous cube.
\subsubsection{Introduce a new faction or type of creature}

A type of creature is a broad grouping: orcs, goblins, lizardmen, the undead, etc.

A faction is a group of creatures united by a similar goal. Once you introduce them you can begin to make moves and cause trouble for the players with those creatures or NPCs.

Introducing means giving some clear sensory evidence or substantiated information. Don't be coy; the players should have some idea what you're showing the presence of. You can, however, be subtle in your approach. No need to have the cultist overlord waving a placard and screaming in the infernal tongue every single time.

A hard application of this move will snowball directly into a combat scene or ambush.
\subsubsection{Use a threat from an existing faction or type of creature}

Once the characters have been introduced to the presence of a faction or type of creature you can use moves of monsters of that type.

Use the factions and types broadly. Orcs are accompanied with their hunting worgs. A mad cult probably has some undead servants or maybe a few beasts summoned from the abyssal pits. This is a move that, often, you'll be making subconsciously--it's just implementing the tools you've set out for yourself in a clear and effective manner.
\subsubsection{Make them backtrack}

Look back at the spaces you've added to the map. Is there anything useful there as yet undiscovered? Can you add a new obstacle that can only be overcome by going back there? Is there a locked door here and now whose key lies in an earlier room?

When backtracking, show the effect that time has had on the areas they've left behind. What new threats have sprung up in their wake? What didn't they take care of that's waiting for their return?

Use this move the make the dungeon a living, breathing place. There is no stasis in the wake of the characters' passing. Add reinforcements, cave in walls, cause chaos. The dungeon evolves in the wake of the characters' actions.
\subsubsection{Present riches at a price}

What do the players want? What would they sacrifice for it?

Put some desirable item just out of reach. Find something they're short on: time, HP, gear, whatever. Find a way to make what they want available if they give up what they have.

The simplest way to use this move is the promise of gold out of the way of the main objective. Will they stop to pry the ruby eyes from the idol when they know that the sacrifice looms closer and closer? Use this move and you can find out.
\subsubsection{Present a challenge to one of the characters}

Challenge a character by looking at what they're good at. Give the thief a lock to pick, show the cleric servants of an enemy god to battle against. Give the wizard magical mysteries to investigate. Show the fighter some skulls to crack. Give someone a chance to shine.

As an alternative, challenge a character by looking at what they're bad at or what they've left unresolved. If the bard has a complicated lie on his conscience, what steps will he take to cover it up when someone figures him out? If the wizard has been summoning demons, what happens when word gets out?

 This move can give a character the spotlight--even if just for a moment. Try to give everyone a chance to be the focus of play using this move from session to session.

\index{Moves!dungeon (GM)|)}
\section*{Dealing With Common Situations}

There are some common situations that come up in Dungeon World. Here's how to deal with them.
\subsubsection{Fights}
\index{GM!running fights}

Sooner or later blades are drawn and blood is shed. When this happens the players are likely to start hacking and slashing, volleying, and defending. Think about more than just the exchange of damage. Monsters might be trying to capture the characters or protect something from them. Understand what the fight is about; what each side wants and how that might affect the tide of battle.

No self-respecting monster just stands still for their beating. Combat is a dynamic thing with creatures moving in and out of range, taking cover, and retreating. Sometimes the battlefield itself shifts. Have your monsters take action that the players will react to. Make sure you're making use of moves beyond deal damage, even in a fight.

Make sure everyone has a chance to act, and that you know where each player is during the chaos of combat. Make a map of a complex battle location so that everyone knows just what's happening and can describe their actions appropriately.
\subsubsection{Traps}
\index{GM!using traps}


Traps may come from your prep, or you can improvise them based on your moves. If nothing has established that the location is safe, traps are always an option.

The players may find traps through clever plans, trap sense, or discerning realities. If a character describes an action that doesn't trigger a move, but the action would still discover a trap, don't hide it from them. Traps aren't allowed to break the rules.
\subsubsection{People}
\index{GM!portraying People}

Dwarven smiths, elven sages, humans of all shapes and sizes occupy the world around the characters. They're not mindless stooges to be pushed around but they're not what we're playing to find out about either. The NPCs are people: they have goals and the tools to struggle towards those goals. Use them to illustrate what the world is like. Show your players the common people struggling for recognition or the noble classes seeking to uplift their people. Some whole adventures might take place in a peopled environment rather than an isolated dungeon. Some classes, the bard in particular, are adept at manipulating and using people as resources. Don't shy away from these situations. Be a fan of these characters, giving them interesting, nuanced people to interact with.

People, just like dungeons, change over time. The passing of the characters through their lives might inspire or enrage them. The characters' actions will cause the world to change, for good or ill, and the people they meet with will remember these changes. When the characters roll back through a town they were less-than-kind to on their previous visit, show them how the people are different now. Are they more cautious? Have they taken up a new religion? Are they hungry for revenge?

Relationships between characters are represented by the bonds but relationships with NPCs are more tenuous. If the players want to make real, lasting connections with the people of the world, they need to act. Remember, ``what do you do?'' is as valid a question when faced with the hopes and fears of a potential new ally or enemy as it is when staring down the business end of a longsword.

 
