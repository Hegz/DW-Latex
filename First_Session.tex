\chapter{First Session}


 The first session of a game of Dungeon World begins with character creation. Character creation is also world creation, the details on the character sheets and the questions that you ask establish what Dungeon World is like--who lives in it and what's going on.


 This section is for the GM so it's addressed to you--the GM. For the players, the first session is just like every other. They just have to play their characters like real people and explore Dungeon World. You have to do a little more in the first session. You establish the world and the threats the players will face.
\section{Prep}


 Before the first session, you'll need to print some stuff. Print off:
\begin{itemize}
\item A few copies of the basic moves
\item One copy of each class sheet, double-sided
\item One copy each of the cleric and wizard spell sheets, double-sided
\item The GM sheet

\end{itemize}


 You'll also need to read this whole book, especially the sections on GMing (GM moves) and the basic moves. It's a good idea to be acquainted with the class moves too, so you can be prepared for them. Be especially sure to read the rules for fronts, but don't create any yet.


 Think about fantastic worlds, strange magic, and foul beasts. Remember the games you played and the stories you told. Watch some movies, read some comics; get heroic fantasy into your brain.


 What you bring to the first session, ideas-wise, is up to you. At the very least bring your head full of ideas. That's the bare minimum.


 If you like you can plan a little more. Maybe think of an evil plot and who's behind it, or some monsters you'd like to use.


 If you've got some spare time on your hands you can even draw some maps (but remember, from your principles: leave blanks) and imagine specific locations.


 The one thing you absolutely can't bring to the table is a planned storyline or plot. You don't know the heroes or the world before you sit down to play so planning anything concrete is just going to frustrate you. It also conflicts with your agenda: play to find out what happens.
\subsection{Getting Started}


 When everyone shows up for the first session briefly introduce Dungeon World to anyone who hasn't played before. Cover the mechanical basis of moves. Introduce the character classes, help players pick their classes, and walk them through character creation.


 Your role during character creation is threefold: help everyone, ask questions, and take notes. When a player makes a choice--particularly for their bonds--ask them about it. Get more detail. Think about what these details mean.


 You should also set expectations: the players are to play their characters as people--skilled adventurers delving into dangerous places, but real people. Your role is to play the rest of the world as a dynamic, changing place.


 Some questions commonly come up during character creation. You should be ready to answer them:


 \emph{Are the characters friends?}
 No, not necessarily, but they do work together as a team for common goals. Their reasons for pursuing those goals may be different, but they manage to work together.


 \emph{Are there other wizards?}
 Not really. There are other workers of arcane magic, and the common folk may call them wizards, but they're not like you. They don't have the same abilities, though they may be similar. Later on there may be another player character with the same class but no GM character will ever really be a wizard (or any other class).


 \emph{What's coin?}
 Coin's the currency of the realm. It's good pretty much everywhere. It'll buy you mundane stuff, like steel swords and wooden staves. The special stuff, like magic weapons, isn't for sale. Not for coin, anyway \ldots 


 \emph{Is the GM trying to kill us?}
 No. The GM's job is to portray the world and the things in it and the world is a very dangerous place. You might die. That doesn't mean the GM is out to get you.


 During this entire process, especially character creation, ask questions. Look for interesting facts established by the characters' bonds, moves, classes, and descriptions and ask about those things. Be curious! When someone mentions the demons that slaughtered their village find out more about them. After all, you don't have anything (except maybe a dungeon) and everything they give you is fuel for future adventures.


 Also pay attention to the players' questions. When mechanical questions come up answer them. When questions of setting or fiction come up your best bet is to turn those questions around. When a player says, ``Who is the King of Torsea,'' say, ``I don't know. Who is it? What is he like?'' Collaborate with your players. Asking a question means it's something that interests them so work with them to make the answers interesting. Don't be afraid to say, ``I don't know'' and ask them the same questions. Work together to find a fantastic and interesting answer.


 If you've come to the table with some ideas about stuff you'd like to see in the world, share them with the players. Their characters are their responsibility and the world is yours--you've got a lot of say in what lives in it. If you want the game to be about a hunt for the lost sorcerer-race of aeons past, say so! If the players aren't interested or they're sick to death of sorcerers, they'll let you know and you can work together to find some other way. You don't need pre-approval for everything but making sure everyone is excited about the broad strokes of the world is a great start.


 Once everyone has their characters created you can take a deep breath. Look back over the questions you've asked and answered so far. You should have some notes that will point you towards what the game might look like. Look at what the players have brought to the table. Look to the ideas that've been stewing away in your head. It's time for the adventure to begin!
\subsection{The First Adventure}


 The first adventure is really about discovering the direction that future sessions will take. Throughout the first adventure keep your eye out for unresolved threats; note dangerous things that are mentioned but not dealt with. These will be fuel for sessions to come.


 Start the session with a group of player characters (maybe all of them) in a tense situation. Use anything that demands action: outside the entrance to a dungeon, ambushed in a fetid swamp, peeking through the crack in a door at the orc guards, or being sentenced before King Levus. Ask questions right away--``who is leading the ambush against you?'' or ``what did you do to make King Levus so mad?'' If the situation stems directly from the characters and your questions, all the better.


 Here's where the game starts. The players will start saying and doing things, which means they'll start making moves. For the first session you should watch especially carefully for when moves apply, until the players get the hang of it. Often, in the early sessions, the players will be most comfortable just narrating their actions--this is fine. When a move triggers let them know. Say, ``It sounds like you're trying to \ldots '' and then walk them through the move. Players looking for direction will look to their character sheet. When a player just says ``I hack and slash him'' be quick to ask, ``so what are you actually doing?'' Ask ``How?'' or ``With what?''


 For the first session, you have a few specific goals:
\begin{itemize}
\item Establish details, describe
\item Use what they give you
\item Ask questions
\item Leave blanks
\item Look for interesting facts
\item Help the players understand the moves
\item Give each character a chance to shine
\item Introduce NPCs

\end{itemize}
\subsubsection{Establish details, describe}


 All the ideas and visions in your head don't really exist in the fiction of the game until you share them, describe them, and detail them. The first session is the time to establish the basics of what things look like, who's in charge, what they wear, what the world is like, what the immediate location is like. Describe everything but keep it brief enough to expand on later. Use a detail or two to make a description really stand out as real.
\subsubsection{Use what they give you}


 The best part of the first session is you don't have to come with anything concrete. You might have a dungeon sketched out but the players provide the real meat--use it. They'll emerge from the darkness of that first dungeon and when they do and their eyes adjust to the light, you'll have built up an exciting world to explore with their help. Look at their bonds, their moves, how they answer your questions and use what you find to fill in the world around the characters.
\subsubsection{Ask questions}


 You're using what they give you, right? What if you need more? That's when you draw it out by asking questions. Poke and prod about specific things. Ask for reactions: ``What does Lux think about that?'' ``Is Avon doing something about it?''


 If you ever find yourself at a loss, pause for a second and ask a question. Ask one character a question about another. When a character does something, ask how a different character feels or reacts. Questions will power your game and make it feel real and exciting. Use the answers you find to fill in what might happen next.
\subsubsection{Leave blanks}


 This is one of your principles, but it's especially true during the first session. Every blank is another cool thing waiting to happen; leave yourself a stock of them.
\subsubsection{Look for interesting facts}


 There are some ideas that, when you hear them, just jump out at you. When you hear one of those ideas, just write it down. When a player mentions the Duke of Sorrows being the demon he bargained with, note it. That little fact is the seed for a whole world.
\subsubsection{Help the players understand the moves}


 You've already read the game, the players may not have, so it's up to you to help them if they need it. The fact is, they likely won't need it much. All they have to do is describe what their character does, the rules take care of the rest.


 The one place they may need some help is remembering the triggers for the moves. Keep an ear out for actions that trigger moves, like attacking in melee or consulting their knowledge. After a few moves the players will likely remember them on their own.
\subsubsection{Give each character a chance to shine}


 As a fan of the heroes (remember your agenda?) you want to see them do what they do best. Give them a chance at this, not by tailoring every room to their skills, but by portraying a fantastic world (agenda again) where there are many solutions to every challenge.
\subsubsection{Introduce NPCs}


 NPCs bring the world to life. If every monster does nothing more than attack and every blacksmith sets out their wares for simple payment the world is dead. Instead give your characters, especially those that the players show an interest in, life (principles, remember?). Introduce NPCs but don't protect them. The recently deceased Lord of Goblins is just as useful for future adventures as the one who's still alive.
\section{After the First Session}


 Once you're done with the first session take some time to relax. Let ideas ferment. Don't rush into the next session.


 Once you've had some time to relax and think over the first session it's time to prepare for the next session. Preparing for the second session takes a few minutes, maybe an hour if this is your first time. You'll create fronts, maybe make some monsters or custom moves, and generally get an idea of what is going on in the world.


