\chapter{Equipment}

The musty tombs and forgotten treasure troves of the world are filled with useful items. The fighter can find a sharp new sword or the thief might stumble across a deadly poison. Most items are mundane--not magical or intrinsically unique in any way. Any item that is magical or one-of-a-kind is not mundane for the purposes of moves. The fighter's signature weapon is never mundane.

Each piece of equipment will have a number of tags. These will tell you something about how the equipment affects the character using it (like +Armor) or suggest something about the way it is used (like the Range tags). Like everything else in Dungeon World, these guide the fiction you're creating in play. If a weapon is awkward, it might mean that you're more likely to drop it when you fail that hack and slash roll.

By no means is this an exhaustive list--feel free to create your own tags.
\subsection{General Equipment Tags}

These are general tags that can apply to just about any piece of gear. You'll see them on armor, weapons or general adventuring tools.

\emph{Applied}
: It's only useful when carefully applied to a person or to something they eat or drink.

\emph{Awkward}
: It's unwieldy and tough to use. 

\emph{+Bonus}
: It modifies your effectiveness in a specified situation. It might be ``+1 forward to spout lore'' or ``-1 ongoing to hack and slash.''

\emph{n coins}
: How much it costs to buy, normally. If the cost includes ``-Charisma'' a little negotiation subtracts the haggler's Charisma score (not modifier) from the price.

\emph{Dangerous}
: It's easy to get in trouble with it. If you interact with it without proper precautions the GM may freely invoke the consequences of your foolish actions.

\emph{Ration}
: It's edible, more or less.

\emph{Requires}
: It's only useful to certain people. If you don't meet the requirements it works poorly, if at all.

\emph{Slow}
: It takes minutes or more to use.

\emph{Touch}
: It's used by touching it to the target's skin.

\emph{Two-handed}
: It takes two hands to use it effectively.

\emph{n weight}
: Count the listed amount against your load. Something with no listed weight isn't designed to be carried. 100 coins in standard denominations is 1 weight. The same value in gems or fine art may be lighter or heavier.

\emph{Worn}
: To use it, you have to be wearing it.

\emph{n Uses}
: It can only be used \emph{n}
times. 
\section*{Weapons}

Weapons don't kill monsters, people do. That's why weapons in Dungeon World don't have a listed damage. A weapon is useful primarily for its tags which describe what the weapon is useful for. A dagger is not useful because it does more or less damage than some other blade. It's useful because it's small and easy to strike with at close distance. A dagger in the hands of the wizard is not nearly so dangerous as one in the hands of a skilled fighter.
\subsection{Weapon Tags}

Weapons may have tags that are primarily there to help you describe them (like Rusty or Glowing) but these tags have a specific, mechanical effect.

\emph{n Ammo}
: It counts as ammunition for appropriate ranged weapons. The number indicated does not represent individual arrows or sling stones, but represents what you have left on hand. 

\emph{Forceful}
: It can knock someone back a pace, maybe even off their feet. 

\emph{+n Damage}
: It is particularly harmful to your enemies. When you deal damage, you add \emph{n}
to it.

\emph{Ignores Armor}
: Don't subtract armor from the damage taken.

\emph{Messy}
: It does damage in a particularly destructive way, ripping people and things apart.

\emph{n Piercing}
: It goes right through armor. When you deal damage with \emph{n}
piercing, you subtract \emph{n}
from the enemy's armor for that attack.

\emph{Precise}
: It rewards careful strikes. You use DEX to hack and slash with this weapon, not STR.

\emph{Reload}
: After you attack with it, it takes more than a moment to reset for another attack. 

\emph{Stun}
: When you attack with it, it does stun damage instead of normal damage. 

\emph{Thrown}
: Throw it at someone to hurt them. If you volley with this weapon, you can't choose to mark off ammo on a 7--9; once you throw it, it's gone until you can recover it. 

Weapons have tags to indicate the range at which they are useful. Dungeon World doesn't inflict penalties or grant bonuses for ``optimal range'' or the like, but if your weapon says Hand and an enemy is ten yards away, a player would have a hard time justifying using that weapon against him.

\emph{Hand}
: It's useful for attacking something within your reach, no further.

\emph{Close}
: It's useful for attacking something at arm's reach plus a foot or two.

\emph{Reach}
: It's useful for attacking something that's several feet away--maybe as far as ten. 

\emph{Near}
: It's useful for attacking if you can see the whites of their eyes. 

\emph{Far}
: It's useful for attacking something in shouting distance.
\subsection{Weapon List}

The stats below are for typical items. There are, of course, variations. A dull long sword might be -1 damage instead while a masterwork dagger could be +1 damage. Consider the following to be stats for typical weapons of their type--a specific weapon could have different tags to represent its features.

{\noindent \bfseries Ragged Bow} \hspace*{\fill} near, 15 coins, 2 weight

{\noindent \bfseries Fine Bow} \hspace*{\fill} near, far, 60 coins, 2 weight

{\noindent \bfseries Hunter's Bow} \hspace*{\fill} near, far, 100 coins, 1 weight

{\noindent \bfseries Crossbow} \hspace*{\fill} near, +1 damage, reload, 35 coins, 3 weight

{\noindent \bfseries Bundle of Arrows} \hspace*{\fill} 3 ammo, 1 coin, 1 weight

{\noindent \bfseries Elven Arrows} \hspace*{\fill} 4 ammo, 20 coins, 1 weight

{\noindent \bfseries Club, Shillelagh} \hspace*{\fill} close, 1 coin, 2 weight

{\noindent \bfseries Staff} \hspace*{\fill} close, two-handed, 1 coin, 1 weight

{\noindent \bfseries Dagger, Shiv, Knife} \hspace*{\fill} hand, 2 coins, 1 weight

{\noindent \bfseries Throwing Dagger} \hspace*{\fill} thrown, near, 1 coin, 0 weight

{\noindent \bfseries Short Sword, Axe, Warhammer, Mace} \hspace*{\fill} close, 8 coins, 1 weight

{\noindent \bfseries Spear} \hspace*{\fill} reach, thrown, near, 5 coins, 1 weight

{\noindent \bfseries Long Sword, Battle Axe, Flail} \hspace*{\fill} close, +1 damage, 15 coins, 2 weight

{\noindent \bfseries Halberd} \hspace*{\fill} reach, +1 damage, two-handed, 9 coins, 2 weight

{\noindent \bfseries Rapier} \hspace*{\fill} close, precise, 25 coins, 1 weight

{\noindent \bfseries Dueling Rapier} \hspace*{\fill} close, 1 piercing, precise, 50 coins, 2 weight
\section*{Armor}

Armor is heavy, difficult to wear and is damned uncomfortable. Some classes are better trained to ignore these drawbacks, but anyone can strap on a suit of armor and enjoy the benefits it grants.
\subsection{Armor Tags}

Armor, like weapons, has tags. Some are purely descriptive but the ones below have some mechanical effect on the player wearing them

\emph{n Armor}
: It protects you from harm and absorbs damage. When you take damage, subtract your armor from the total. If you have more than one item with \emph{n Armor}
, only the highest value counts. 

\emph{+n Armor}
: It protects you and stacks with other armor. Add its value to your total armor.

\emph{Clumsy}
: It's tough to move around with. -1 ongoing while using it. This penalty is cumulative.
\subsection{Armor List}

{\noindent \bfseries Leather, Chainmail} \hspace*{\fill} 1 armor, worn, 10 coins, 1 weight

{\noindent \bfseries Scale Mail} \hspace*{\fill} 2 armor, worn, clumsy, 50 coins, 3 weight

{\noindent \bfseries Plate} \hspace*{\fill} 3 armor, worn, clumsy, 350 coins, 4 weight

{\noindent \bfseries Shield} \hspace*{\fill} +1 armor, 15 coins, 2 weight
\section*{Dungeon Gear}

{\noindent \bfseries Adventuring Gear} \hspace*{\fill} 5 uses, 20 coins, 1 weight

Adventuring gear is a collection of useful mundane items such as chalk, poles, spikes, ropes, etc. When you rummage through your adventuring gear for some useful mundane item, you find what you need and mark off a use.

{\noindent \bfseries Bandages} \hspace*{\fill} 3 uses, slow, 5 coins, 0 weight

When you have a few minutes to bandage someone else's wounds, heal them of 4 damage and expend a use.

{\noindent \bfseries Poultices and Herbs} \hspace*{\fill} 2 uses, slow, 10 coins, 1 weight

When you carefully treat someone's wounds with poultices and herbs, heal them of 7 damage and expend a use.

{\noindent \bfseries Healing Potion} \hspace*{\fill} 50 coins, 0 weight

When you drink an entire healing potion, heal yourself of 10 damage or remove one debility, your choice.

{\noindent \bfseries Keg of Dwarven Stout} \hspace*{\fill} 10 coins, 4 weight

When you open a keg of dwarven stout and let everyone drink freely, take +1 to your Carouse roll. If you drink a whole keg yourself, you are very, very drunk.

{\noindent \bfseries Bag of Books} \hspace*{\fill} 5 uses, 10 coins, 2 weight

When your bag of books contains just the right book for the subject you're spouting lore on, consult the book, mark off a use, and take +1 to your roll.

{\noindent \bfseries Antitoxin} \hspace*{\fill} 10 coins, 0 weight

When you drink antitoxin, you're cured of one poison affecting you.

{\noindent \bfseries Dungeon Rations} \hspace*{\fill} Ration, 5 uses, 3 coins, 1 weight

Not tasty, but not bad either.

{\noindent \bfseries Personal Feast} \hspace*{\fill} Ration, 1 use, 10 coins, 1 weight

Ostentatious to say the least.

{\noindent \bfseries Dwarven Hardtack} \hspace*{\fill} Requires Dwarf, ration, 7 uses, 3 coins, 1 weight

Dwarves say it tastes like home. Everyone else says it tastes like home, if home is a hog farm, and on fire.

{\noindent \bfseries Elven Bread} \hspace*{\fill} Ration, 7 uses, 10 coins, 1 weight

Only the greatest of elf-friends are treated to this rare delicacy.

{\noindent \bfseries Halfling Pipeleaf} \hspace*{\fill} 6 uses, 5 coins, 0 weight

When you share halfling pipeleaf with someone, expend two uses and take +1 forward to parley with them.
\section*{Poisons}

{\noindent \bfseries Oil of Tagit} \hspace*{\fill} Dangerous, applied, 15 coins, 0 weight

The target falls into a light sleep.

{\noindent \bfseries Bloodweed} \hspace*{\fill} Dangerous, touch, 12 coins, 0 weight

Until cured, whenever the afflicted rolls damage, they roll an additional d4 and subtract that result from their normal damage.

{\noindent \bfseries Goldenroot} \hspace*{\fill} Dangerous, applied, 20 coins, 0 weight

The target treats the next creature they see as a trusted ally, until proved otherwise.

{\noindent \bfseries Serpent's Tears} \hspace*{\fill} Dangerous, touch, 10 coins, 0 weight

Anyone dealing damage against the target rolls twice and takes the better result.
\section*{Services}

{\noindent \bfseries A week's stay at a peasant inn} \hspace*{\fill} 14-Charisma coins

{\noindent \bfseries A week's stay at a civilized inn} \hspace*{\fill} 30-Charisma coins

{\noindent \bfseries A week's stay at the fanciest inn in town} \hspace*{\fill} 43-Charisma coins

{\noindent \bfseries A week's unskilled mundane labor} \hspace*{\fill} 10 coins

{\noindent \bfseries A month's pay for enlistment in an army} \hspace*{\fill} 30 coins

{\noindent \bfseries A custom item from a blacksmith} \hspace*{\fill} Base Item + 50 coins

{\noindent \bfseries A night's ``companionship''} \hspace*{\fill} 20-Charisma coins

{\noindent \bfseries An evening of song and dance} \hspace*{\fill} 18-Charisma coins

{\noindent \bfseries Escort for a day along a bandit-infested road} \hspace*{\fill} 20 coins

{\noindent \bfseries Escort for a day along a monster-infested road} \hspace*{\fill} 54 coins

{\noindent \bfseries A run-of-the-mill killing} \hspace*{\fill} 5 coins

{\noindent \bfseries An assassination} \hspace*{\fill} 120 coins

{\noindent \bfseries Healing from a chirurgeon} \hspace*{\fill} 5 coins

{\noindent \bfseries A month's prayers for the departed} \hspace*{\fill} 1 coin

{\noindent \bfseries Repairs to a mundane item} \hspace*{\fill} 25\% of the item's cost
\section*{Meals}

{\noindent \bfseries A hearty meal for one} \hspace*{\fill} 1 coin

{\noindent \bfseries A poor meal for a family} \hspace*{\fill} 1 coin

{\noindent \bfseries A feast} \hspace*{\fill} 15 coins per person
\section*{Transport}

{\noindent \bfseries Cart and Donkey, sworn to carry your burdens} \hspace*{\fill} 50 coins, load 20

{\noindent \bfseries Horse} \hspace*{\fill} 75 coins, load 10

{\noindent \bfseries Warhorse} \hspace*{\fill} 400 coins, load 12

{\noindent \bfseries Wagon} \hspace*{\fill} 150 coins, load 40

{\noindent \bfseries Barge} \hspace*{\fill} 50 coins, load 15

{\noindent \bfseries River boat} \hspace*{\fill} 150 coins, load 20

{\noindent \bfseries Merchant ship} \hspace*{\fill} 5,000 coins, load 200

{\noindent \bfseries War ship} \hspace*{\fill} 20,000 coins, load 100

{\noindent \bfseries Passage on a safe route} \hspace*{\fill} 1 coin

{\noindent \bfseries Passage on a tough route} \hspace*{\fill} 10 coins

{\noindent \bfseries Passage on a dangerous route} \hspace*{\fill} 100 coins
\section*{Land and Buildings}

{\noindent \bfseries A hovel} \hspace*{\fill} 20 coins

{\noindent \bfseries A cottage} \hspace*{\fill} 500 coins

{\noindent \bfseries A house} \hspace*{\fill} 2,500 coins

{\noindent \bfseries A mansion} \hspace*{\fill} 50,000 coins

{\noindent \bfseries A keep} \hspace*{\fill} 75,000 coins

{\noindent \bfseries A castle} \hspace*{\fill} 250,000 coins

{\noindent \bfseries A grand castle} \hspace*{\fill} 1,000,000 coins

{\noindent \bfseries A month's upkeep} \hspace*{\fill} 1\% of the cost
\section*{Bribes}

{\noindent \bfseries A peasant dowry} \hspace*{\fill} 20-Charisma coins

{\noindent \bfseries ``Protection'' for a small business} \hspace*{\fill} 100-Charisma coins

{\noindent \bfseries A government bribe} \hspace*{\fill} 50-Charisma coins

{\noindent \bfseries A compelling bribe} \hspace*{\fill} 80-Charisma coins

{\noindent \bfseries An offer you can't refuse} \hspace*{\fill} 500-Charisma coins
\section*{Gifts and Finery}

{\noindent \bfseries A peasant gift} \hspace*{\fill} 1 coin

{\noindent \bfseries A fine gift} \hspace*{\fill} 55 coins

{\noindent \bfseries A noble gift} \hspace*{\fill} 200 coins

{\noindent \bfseries A ring or cameo} \hspace*{\fill} 75 coins

{\noindent \bfseries Finery} \hspace*{\fill} 105 coins

{\noindent \bfseries A fine tapestry} \hspace*{\fill} 350+ coins

{\noindent \bfseries A crown fit for a king} \hspace*{\fill} 5,000 coins
\section*{Hoards}

{\noindent \bfseries A goblin's stash} \hspace*{\fill} 2 coins

{\noindent \bfseries A lizardman's trinkets} \hspace*{\fill} 5 coins

{\noindent \bfseries A ``priceless'' sword} \hspace*{\fill} 80 coins

{\noindent \bfseries An orc warchief's tribute} \hspace*{\fill} 250 coins

{\noindent \bfseries A dragon's mound of coins and gems} \hspace*{\fill} 130,000 coins
\section*{Magic Items}

There are stranger things in the world than swords and leather. Magic items are the non-mundane items that have intrinsic power.

Magic items are for you to make for your game. Players can make magic items through the wizard's ritual and similar moves. The GM can introduce magic items in the spoils of battle or the rewards for jobs and quests. This list provides some ideas, but magic items are ultimately for you to decide.

When making your own magic items keep in mind that these items are \emph{magical}. Simple modifiers, like+1 damage, are the realm of the mundane--magic items should provide more interesting bonuses.

{\noindent \bfseries Argo-Thaan, Holy Avenger} \hspace*{\fill} Close, 2 weight

There are many swords in this world, but there is only one Argo-thaan. It is a blade of gold, silver and light, revered as a holy relic by all orders and religions for whom Good rings true. Its touch is a blessing and to many, the sight of it brings tears of joy.

In the hands of a paladin, it strikes true and strong. A paladin wielding it increases their damage die to d12 and has access to every paladin move. As well, Argo-thaan can harm any creature of Evil, regardless of any defenses it may have. No Evil creature may touch it without suffering agony. In the hands of any non-paladin, it is merely a sword, heavier and more cumbersome than most--it gains the awkward tag.

Argo-thaan, while not intelligent, will forever be drawn to a cause of true Good, like iron to a magnet.

{\noindent \bfseries Arrows of Acheron} \hspace*{\fill} 1 ammo, 1 weight

Crafted in darkness by a blind fletcher, these arrows can find their target in even the deepest darkness. An archer may fire them blind, in the dark, with his eyes bound by heavy cloth and still be assured of a clean shot. If the light of the sun ever touches the arrows, however, they come apart like shadows and dust.

{\noindent \bfseries Axe of the Conqueror-King} \hspace*{\fill} Close, 1 weight

It is crafted of shining steel, glowing with a golden light and imbued with mythical powers of authority. When you bear the axe, you become a beacon of inspiration to all you lead. Any hirelings in your employ have +1 Loyalty, no matter the quality of your leadership.

{\noindent \bfseries Barb of the Black} \hspace*{\fill} Gate 0 weight

A nail or spike, twisted and forever cold, said to have been pried from the Gates of Death. When hammered into a corpse, it disappears and ensures that corpse will never be risen again--no magic short of that of Death himself can reignite the flame of life (natural or otherwise) in the body.

{\noindent \bfseries Bag of Holding} \hspace*{\fill} 0 weight

A bag of holding is larger on the inside than the outside, it can contain an infinite number of items, and its weight never increases. When you try to retrieve an item from a bag of holding, roll+WIS. *On a 10+, it's right there. *On a 7-9, choose one:
\begin{itemize}
\item You get the exact item, but it takes a while
\item You get a similar item of the GM's choice, but it only takes a moment

\end{itemize}

No matter how many items it contains, a bag of holding is always 0 weight.

{\noindent \bfseries The Burning Wheel} \hspace*{\fill} 2 weight

An ancient wooden wheel, as might appear on a war-wagon, banded with steel. On a glance, it appears to be nothing special--many spokes are shattered and the thing seems mundane. Under the scrutiny of magic or the eyes of an expert, its true nature is revealed: the Burning Wheel is a gift from the God of Fire and burns with his authority.

When you hold The Burning Wheel and speak a god's name, roll+CON. *On a 7+, the god you name takes notice and grants you an audience. An audience with a god is not without a price: on a 10+, you choose one of your stats and reduce it to the next lowest modifier (for example, a 14 is +1, so it would be reduced to 12, a +0). *On a 7--9, the GM chooses which stat to reduce.

Once used, the Burning Wheel ignites and burns with brilliant light. It does not confer any protection from those flames, nor does it provide any bonus to swimming.

{\noindent \bfseries Captain Bligh's Cornucopia} \hspace*{\fill} 1 weight

A brass naval horn, curled and ornate, carved with symbols of the gods of Plenty. When blown, in addition to sound, the horn spills forth food. Enough to feed a meal to everyone who hears its sound.

{\noindent \bfseries The Carcosan Spire} \hspace*{\fill} Reach, Thrown, 3 weight

None know from whence this spear of twisted white coral comes. Those who bear it too long find their minds full of alien dreams and begin to hear the strange thoughts of the Others. None are impervious. Used against any ``natural'' target (men, goblins, owlbears and the like) the Spire acts as a mere mortal spear. Its true purpose is to do harm to those things whose strange natures protect them against mundane weapons. Used thus, the Spire can wound foes otherwise invulnerable to harm. The wielder will recognize these twisted foes on sight--the Spire knows its own.

{\noindent \bfseries Cloak of Silent Stars} \hspace*{\fill} 1 weight

A cape of rich black velvet outside and sparkling with tiny points of light within, this cloak bends fate, time and reality around it to protect the wearer, who may defy danger with whatever stat they like. To do this, the wearer invokes the cloak's magic and their player describes how the cloak helps ``break the rules.'' They can deflect a fireball with CHA by convincing it they deserve to live or elude a fall by applying the mighty logic of their INT to prove the fall won't hurt. The cloak makes it so. It can be used once for each stat before losing its magic.

{\noindent \bfseries Coin of Remembering} \hspace*{\fill} 0 weight

What appears, at a glance, to be a simple copper coin is, in truth, an enchanted coin. Its bearer can, at any time, redeem it to know immediately one fact that has been forgotten. The coin vanishes thereafter. It does not have to be a thing forgotten by the bearer, but it cannot be ``known.'' Interpretation of this stipulation is left to the gods. If the coin is unsuccessful, it will still paint an image in the mind's eye of someone or something that does remember what was sought.

{\noindent \bfseries Common Scroll} \hspace*{\fill} 1 use, 0 weight

A common scroll has a spell inscribed on it. The spell must be castable by you or on your class's spell list for you to be able to cast it. When you cast a spell from a scroll, the spell takes effect, simple as that.

{\noindent \bfseries Devilsbane Oil} \hspace*{\fill} 1 use, 0 weight

A holy oil, created in limited supply by a mute sect of mountain monks whose order protected humanity from the powers of the Demon Pits in ancient epochs. Only a few jars remain. When applied to any weapon and used to strike a denizen of any outer plane, the oil undoes the magic that binds that creature. In some cases, this will return it to its home. In others, it merely undoes any magic controlling it. The oil stays on the weapon for a few hours before it dries and flakes away.

If applied to the edges of a doorway or drawn in a circle, the oil will repel creatures whose home is any of the outer planes. They cannot pass across it. The oil lasts for one full day before it soaks in or evaporates.

{\noindent \bfseries Earworm Wax} \hspace*{\fill} 1 use, 0 weight

A yellowish candle. Seems never to burn out and the light it casts is strange and weak. Its wax is always cool, too. Drip the wax into the ear of a target and gain 3 hold. Spend that hold and ask your target a question. They find themselves telling you the whole truth, despite themselves. The consequences, after the fact? Those are up to you to deal with.

{\noindent \bfseries The Echo} \hspace*{\fill} 0 weight

A seemingly empty bottle. Once unstoppered, the whispers of another plane resound once and fall silent. In the silence, the bearer learns in his soul the coming of one great danger and how he can avoid it. At any point after you use the Echo, you can ignore the results of any single die roll--yours or another player's--and roll again. Once opened, the Echo is released and gone forever.

{\noindent \bfseries The Epoch Lens} \hspace*{\fill} 1 weight

An archmage, old and too frail to leave his tower, crafted this intricate and fragile device of glass and gold to examine the histories and relics he so loved. Looking at an object through the lens reveals visions of who made it and where it came from.

{\noindent \bfseries Farsight Stone} \hspace*{\fill} 1 weight

Swirling clouds fill this smoky orb and those in its presence often hear strange whispers. In ancient times, it was part of a network of such stones, used to communicate and surveil across great distances. When you gaze into the stone, name a location and roll+WIS. *On a 10+, you see a clear vision of the location and can maintain it as long as you concentrate on the orb. *On a 7--9, you still see the vision, but you draw the attention of some other thing (an angel, a demon, or the holder of another Farsight stone) that uses the stone to surveil you, as well.

{\noindent \bfseries The Fiasco Codex} \hspace*{\fill} 0 weight

A thick tome, said to be penned in the blood of poor fools and robber-barons by some demon prince possessed of dark humor, this tome details tales and stories of those whose ambition overwhelmed their reason. Reading from this tome teaches one the value of clear-headedness but leaves a sense of dread behind. When you read from the Fiasco Codex, Roll+WIS. *On a 10+, ask two of the questions below. *On a 7--9, ask one.
\begin{itemize}
\item What is my greatest opportunity, right now?
\item Who can I betray to gain an advantage?
\item Who is an ally I should not trust?

\end{itemize}

The codex gives up its answers only once to each reader and takes 2 to 3 hours to read.

{\noindent \bfseries Flask of Breath} \hspace*{\fill} 0 weight

A simple thing, but useful when you need a breath of fresh air. The flask appears empty but cannot be filled, anything added to it simply spills out. This is because the flask is eternally full of air. If placed underwater, it will bubble forever. If pressed to the mouth, one can breathe normally--smoke is no concern, for example. I'm sure you'll find all sorts of unusual uses for it.

{\noindent \bfseries Folly Held Aloft, The Wax Wings, A Huge Mistake} \hspace*{\fill} 1 weight

Who hasn't always wanted to soar the pretty blue sky? In an attempt to grant the wishes of land-bound folk, these great magical wings were created. Known by many names and crafted by as many mages, they commonly take the shape of the wings of whatever local birds hold affection. Worn by means of a harness or, in some dire cases, a surgical procedure.

When you take to the air with these magical wings, roll+DEX. *On a 10+, your flight is controlled and you may stay aloft as long as you like. *On a 7--9, you make it aloft but your flight is short or erratic and unpredictable, your choice. *On a 6-, you make it aloft, but the coming-down part and everything between is up to the GM.

{\noindent \bfseries Immovable Rod} \hspace*{\fill} 0 weight

A funny metal rod with a button on it. Press the button and the rod just sticks. It freezes in place--in midair, standing up or lying down. It can't be moved. Pull it, push it, try as hard as you like, the rod stays. Maybe it can be destroyed, maybe it can't. Push the button again and it's free--take it along with you. Might be useful to have such a stubborn thing along. 

{\noindent \bfseries Infinite Book} \hspace*{\fill} 1 weight

This book contains an infinite number of pages in a finite space. With no limit to the pages, everything that ever was, is, or will be is contained somewhere in the book. Luckily the index is great.

When you spout lore while consulting the book you gain an extra clause: On a 12+, the GM will give a solution to a problem or situation you're in.

{\noindent \bfseries Inspectacles} \hspace*{\fill} 0 weight

Rough-hewn glass in wooden frames. Dinged up and barely held together, they somehow allow the wearer to see much more than their naked eyes might. When you discern realities wearing these gifted lenses, you get to bend the rules a little. On a roll of 10+, ask any three questions you like. They don't have to be on the list. As long as sight could give you answers, the GM will tell you what you want to know.

{\noindent \bfseries The Ku'meh Maneuver} \hspace*{\fill} 1 weight

A great, leathery tome worn shiny by the hands of a hundred great generals, this book is often passed from warrior to warrior, from father to son along the great battle lines that have divided Dungeon World's past. Anyone reading it may, upon finishing for the first time, roll+INT. *On a 10+, hold 3. *On a 7-9, hold 1. You may spend your hold to advise a companion on some matter of strategic or tactical significance. This advice allows you to, at any time, regardless of distance, roll to aid them on any one roll. On a miss, the GM can hold 1 and spend it to apply -2 to any roll of yours or the poor sap who listened to your advice.

{\noindent \bfseries Lamented Memento} \hspace*{\fill} 0 weight

Taking the form of a single lock of bright red hair, bound in a black ribbon and immune to the ravages of time, the Lamented Memento bears a grim enchantment. In it are the memories and emotions of a girl who dealt with Death at the Black Gates so many times that, in the end, they fell in love and she left the world to be with him for a time. Her memory protects the wielder. If he finds himself at the Gates, the Memento can be traded for an automatic result of 10+ on the Last Breath move.

{\noindent \bfseries Lodestone Shield} \hspace*{\fill} +1 armor, 1 weight

What mixed-up dummy made this? Shields are meant to repel metal, not draw it in! Emblazoned with a lion rampant, the Lodestone Shield has the power to pull blades and arrows to it. When you defend against enemies using metal weapons you can spend one hold, per target, to disarm them. Also, sometimes you'll find a handful of loose change stuck to it.

{\noindent \bfseries Map of the Last Patrol} \hspace*{\fill} 0 weight

An ancient order of brave rangers once patrolled the land, protecting villages and warning kings and queens of encroaching danger. They're long gone, now, but their legacy remains. This map, when marked with the blood of a group of people, will always show their location--so long as they remain within the bounds of the map. 

{\noindent \bfseries Ned's Head} \hspace*{\fill} 1 weight

An old skull, missing its jaw and very much worse-for-wear. The skull remembers the folly of its former owner--a man with more honor than sense. Once per night, the owner of the skull can ask ``Who has it in for me?'' and the skull will give up one name in a sad, lonely voice. If the owner of the skull is ever killed, it disappears surreptitiously. No one knows where it might turn up next.

{\noindent \bfseries Nightsider's Key} \hspace*{\fill} 0 weight

This key unlocks any door for you, provided you don't belong where you intend to go. So long as you do nothing that would alert another to your presence (remaining unheard, unseen and unnoticed) and takes nothing more than your memories out with you, the key's magic will prevent your intrusion from ever being discovered. It's like you were never there at all.

{\noindent \bfseries Sacred Herbs} \hspace*{\fill} 0 weight

The sacred herbs, collected and prepared by an order of lost wizard-monks, can be found in bundles with two or three uses to them. Kept dry, they last indefinitely. When smoked in a pipe or consumed in an incense burner and the thick, blue smoke inhaled, these herbs will grant you strange visions of faraway places and distant times. If you focus your will on a particular person, place or thing, the herbs will respond: roll+WIS. *On a 10+, the vision is clear and useful--yielding some valid information. *On a 7--9, the vision is about the thing desired, but is unclear, fraught with metaphor or somehow difficult to understand. *On a miss, the GM will ask you, ``What is it you fear most?'' You must answer honestly, of course.

{\noindent \bfseries The Sartar Duck} \hspace*{\fill} 0 weight

An odd, hand-carved wooden duck. Who would make such a funny thing? While you bear it, you find yourself an exceptionally gifted storyteller--no matter the language, you can make yourself and your story clear to any audience. They will understand your meaning, if not your words.

{\noindent \bfseries Tears of Annalise} \hspace*{\fill} 0 weight

Cloudy red gemstones the size of a thumbnail, the Tears of Annalise are always found in pairs. When swallowed by two different people, they bind the swallowers together--when either feels strong emotions (particularly sadness, loss, fear or desire) the other feels it, as well. The effects last until one spills the blood of the other.

{\noindent \bfseries Teleportation Room} \hspace*{\fill} Slow

James Ninefingers, eccentric genius mage, created these room-sized magical apparati. A stone chamber etched with runes and scribblings, glowing with a faint blue light. When you enter and say aloud the name of a location, roll+INT. *On a 10+, you arrive exactly where you'd intended. *On a 7--9, the GM chooses a safe location nearby. *On a miss, you end up someplace. Maybe it's nearby? It's definitely not safe. Strange things sometimes happen to those who bend time and space with these devices.

{\noindent \bfseries Timunn's Armor} \hspace*{\fill} 1 armor, 1 weight

A stealthy suit of armor, it appears as many things to many people and blends in with appropriate apparel. The wearer always seems the height of fashion to any who gaze upon him.

{\noindent \bfseries Titus' Truthful Tallow} \hspace*{\fill} 0 weight

A candle of ivory- and copper-colored tallow with a wick of spun silver. When lit, none upon whom its light falls is able to tell a lie. They may keep silent or dissemble but when asked a question directly, they can speak naught but truth.

{\noindent \bfseries Tricksy Rope} \hspace*{\fill} 1 weight

A rope that listens. Does tricks, too, like a smart and more obedient snake might. Tell it ``Coil'' or ``Slack'' or ``Come here, rope'' and it will. 

{\noindent \bfseries The Sterling Hand} \hspace*{\fill} 0 weight

Crafted by dwarven whitesmiths, this mirrored-metal hand is deeply scored with runes of power and rejuvenation. Meant to replace wounded or destroyed limbs from mining accidents, the Sterling Hand bonds to the wound, old or new, and is strong and stout. It can be used as a weapon (Near range) and is made of pure enough silver to harm creatures affected by such.

{\noindent \bfseries Vellius's Gauntlets} \hspace*{\fill} 1 weight

Crafted in the name of Vellius the Clumsy, Vellius the Butter-Fingered, Vellius the Clod, these gloves of simple cloth prevent you from dropping any object you don't intent to. You cannot be disarmed and will not fall from any rope or ladder, for example. This item can get very messy if you have something strong pulling at your legs while you grip onto something solid.

{\noindent \bfseries Violation Glaive} \hspace*{\fill} Reach, 2 weight

A legendary blade, said to have been thrust backwards in time from some grim future, the violation glaive is crafted of strange green iron. The blade strikes at the mind of those it wounds, as well as the body. When you hack and slash on a 10+ you have an additional option: you can deal your normal damage, let them counterattack you, and instill the emotion of your choice (maybe fear, reverence, or trust).

{\noindent \bfseries Vorpal Sword} \hspace*{\fill} Close, 3 piercing, 2 weight

Snicker-snack and all that. Sharp as anything, this simple-seeming sword means to separate one thing from another--the limb from the body or folk from their lives. When you deal damage with the Vorpal Sword, your enemy must choose something (an item, an advantage, a limb) and lose it, permanently.


