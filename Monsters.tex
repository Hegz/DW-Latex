\chapter{Monsters}


 Great heroes need horrendous antagonists. This section is about how to create and play as those antagonists--from the lowly goblin warrior to the hellish demon.
\section*{Using Monsters}


 A monster is any living (or undead) thing that stands in the characters' way.


 How you use these monsters follows directly from your agenda and principles. Stay true to your principles, use your moves and pursue your agenda--you can't go wrong. 


Your first agenda is to \emph{portray a fantastic world}. The way you describe the monsters and adversaries the characters face can be a tool to help you fulfill that agenda. Describing those creatures and people in vivid detail will bring them to life. You'll eventually need some stats for those monsters, too--the rules in this chapter are here to help you create those stats quickly and easily.


 The player characters are the heroes. Monsters exist to illustrate what a dangerous awful place Dungeon World can be--how it will remain if the heroes don't step in. You shouldn't be rooting for the monsters to win, but they may challenge, and even sometimes defeat, the heroes. If you feel like your monsters are being beaten too quickly, don't worry. Let the players revel in their victory, there's always more monsters.


 The principle of \emph{think dangerous}
 sums up that philosophy--think of every monster like an arrow fired at the characters. The monsters are ammunition of the danger you're presenting. Some may be smarter, faster, or more dangerous than others but until a monster warrants a name, a personality, or some other special consideration, it's an arrow. Take aim and shoot. Don't worry if you miss.


 A monster stops being mere ammunition when something in the world puts the spotlight on it. This might be a spout lore roll that leads your description in an interesting direction or the result of your asking questions and building on the players' answers. Maybe the characters were overwhelmed in battle and had to run away, giving them a new-found fear and respect for the beast they fought. When these things happen, feel free to give the monster a name and consider creating a danger to represent it.


 One thing that your agenda and principles don't say anything about is setting up a fair fight. Heroes are often outnumbered or faced with ridiculous odds--sometimes they have to retreat and make a new plan. Sometimes they suffer loss. When adding a monster to a front, placing them in a dungeon, or making them up on the fly your first responsibility is to the fiction (\emph{portray a fantastic world}
) and to give the characters a real threat (\emph{make the characters heroes}
), not to make a balanced fight. Dungeon World isn't about balancing encounter levels or counting experience points; it's about adventure and death-defying feats!
\section*{Elements of a Monster}


 Every monster has \textbf{moves}
 that describe its behavior and abilities. Just like the normal GM moves, they're things that you do when there's a lull in the action or when the players give you a golden opportunity. As with other GM moves they can be hard or soft depending on the circumstances and the move: a move that's irreversible and immediate is hard, a move that's impending or easy to counter is soft.


 Each monster has an \textbf{instinct}
 that describe its goals at a high level. Some monsters live for conquest, or treasure, or simply for blood. The monster's instinct is the guide to how to use it in the fiction.


 The monster's \textbf{description}
 is where all its other features come from. The description is how you know what the monster really is, the other elements just reflect the description.


 \textbf{Damage}
 is a measure of how much pain the monster can inflict at once. Just like player damage it's a die to roll, maybe with some modifiers. A monster deals its damage to another monster or a player whenever it causes them physical harm.


 Each monster has \textbf{tags}
 that describe how it deals damage, including the range(s) of its attacks. When trying to attack something out of its range (too close or too far) the monster's out of luck, no damage. Any tag that can go on a weapon (like messy or slow) can also go on a monster.


 There are special tags that apply only to monsters. These tags, listed below, describe the key attributes of the monster--qualities that describe how big they are and how, if at all, they organize themselves.


 A monster's \textbf{HP}
 is a measure of how much damage it can take before it dies. Just like players, when a monster takes damage it subtracts that amount from its HP\@. At 0 HP it's dead, no last breath.


 Some monsters are lucky enough to enjoy \textbf{armor}. It's just like player armor: when a monster with armor takes damage it subtracts its armor from the damage done.


 \textbf{Special qualities}
 describe innate aspects of the monster that are important to play. These are a guide to the fiction, and therefore the moves. A quality like intangible means just what it says: mundane stuff just passes through it. That means swinging a mundane sword at it isn't hack and slash, for a start.
\subsection{Monsters Without Stats}


 Some creatures operate on a scale so far beyond the mortal that concepts like HP, armor, and damage just do not hold. Some creatures just aren't dangerous in a fight. These creatures may still cause problems for the players and may even be defeated with clever thinking and enough preparation.


 If a creature is of such a scale far beyond the players, or if it just doesn't put up a physical fight, don't assign it HP, damage, or armor. You can still use the monster creation rules to give it tags. The core of a stat-less monster is its instinct and moves; you can have it make its moves and act according to its instinct even without numeric stats.
\section*{Monster Tags}


 \emph{Magical}
: It is by nature magical through and through.


 \emph{Devious}
: Its main danger lies beyond the simple clash of battle.


 \emph{Amorphous}
: Its anatomy and organs are bizarre and unnatural.


 \emph{Organized}
: It has a group structure that aids it in survival. Defeating one may cause the wrath of others. One may sound an alarm.


 \emph{Intelligent}
: It's smart enough that some individuals pick up other skills. The GM can adapt the monster by adding tags to reflect specific training, like a mage or warrior.


 \emph{Hoarder}
: It almost certainly has treasure.


 \emph{Stealthy}
: It can avoid detection and prefers to attack with the element of surprise.


 \emph{Terrifying}
: Its presence and appearance evoke fear.


 \emph{Cautious}
: It prizes survival over aggression.


 \emph{Construct}
: It was made, not born


 \emph{Planar}
: It's from beyond this world
\subsection{Organization Tags}


 \emph{Horde}
: Where there's one, there's more. A lot more.


 \emph{Group}
: Usually seen in small numbers, 3--6 or so.


 \emph{Solitary}
: It lives and fights alone.
\subsection{Size Tags}


 \emph{Tiny}
: It's much smaller than a halfling.


 \emph{Small}
: It's about halfling size.


 \emph{Large}
: It's much bigger than a human, about as big as a cart.


 \emph{Huge}
: It's as big as a small house or larger.
\section*{Making Monsters}


 Monsters start with your description of them. Whether you're making the monster before play or just as the players come face-to-face with it, every monster starts with a clear vision of what it is and what it does.


 If you're making a monster between sessions start by imagining it. Imagine what it looks like, what it does, why it stands out. Imagine the stories told about it and what effects it has had on the world.


 If you're making a monster on the fly during a session start by describing it to the players. Your description starts before the characters even lay eyes on it: describe where it lives, what marks it has made on the environment around it. Your description is the key to the monster.


 When you find you need stats for the monster you use this series of questions to establish them. Answer every question based on the facts established and imagined. Don't answer them aloud to anyone else, just note down the answers and the stats listed with each answer.


 If two questions would grant the same tag don't worry about it. If you like you can adjust damage or HP by 2 to reflect the tag that would be repeated, but it's not necessary. If a combination of answers would reduce HP or damage below 1 they stay at 1.


 When you're finished your monster may have only one move. If this is the case and you plan on using the monster often, give it another 1--2 moves of your choice. These moves often describe secondary modes of attack, other uses for a primary mode of attack, or connections to a certain place in the world.
\subsubsection{What is it known to do?}


 Write a monster move describing what it does.
\subsubsection{What does it want that causes problems for others?}


 This is its instinct. Write it as an intended action.
\subsubsection{How does it usually hunt or fight?}
\begin{itemize}
\item In large groups: horde, d6 damage, 3 HP
\item In small groups, about 2--5: group, d8 damage, 6 HP
\item All by its lonesome: solitary, d10 damage, 12 HP

\end{itemize}
\subsubsection{How big is it?}
\begin{itemize}
\item Smaller than a house cat: tiny, hand, -2 damage
\item Halfling-esque: small, close
\item About human size: close
\item As big as a cart: large, close, reach, +4 HP, +1 damage
\item Much larger than a cart: huge, reach, +8 HP, +3 damage

\end{itemize}
\subsubsection{What is its most important defense?}
\begin{itemize}
\item Cloth or flesh: 0 armor
\item Leathers or thick hide: 1 armor
\item Mail or scales: 2 armor
\item Plate or bone: 3 armor
\item Permanent magical protection: 4 armor, magical

\end{itemize}
\subsubsection{What is it known for? (Choose all that apply)}
\begin{itemize}
\item Unrelenting strength: +2 damage, forceful
\item Skill in offense: roll damage twice and take the better roll
\item Skill in defense: +1 armor
\item Deft strikes: +1 piercing
\item Uncanny endurance: +4 HP
\item Deceit and trickery: stealthy, write a move about dirty tricks
\item A useful adaptation like being amphibious or having wings: Add a special quality for the adaptation
\item The favor of the gods: divine, +2 damage or +2 HP or both (your call)
\item Spells and magic: magical, write a move about its spells

\end{itemize}
\subsubsection{What is its most common form of attack?}


 Note it along with the creature's damage. Common answers include: a type of weapon, claws, a specific spell. Then answer these questions about it:
\begin{itemize}
\item Its armaments are vicious and obvious: +2 damage
\item It lets the monster keep others at bay: reach
\item Its armaments are small and weak: reduce its damage die size by one
\item Its armaments can slice or pierce metal: messy, +1 piercing or +3 piercing if it can just tear metal apart
\item Armor doesn't help with the damage it deals (due to magic, size, etc.): ignores armor
\item It usually attacks at range (with arrows, spells, or other projectiles): near or far or both (your call)

\end{itemize}
\subsubsection{Which of these describe it? (Choose all that apply)}
\begin{itemize}
\item It isn't dangerous because of the wounds it inflicts, but for other reasons: devious, reduce its damage die size by one, write a move about why it's dangerous
\item It organizes into larger groups that it can call on for support: organized, write a move about calling on others for help
\item It's as smart as a human or thereabouts: intelligent
\item It actively defends itself with a shield or similar: cautious, +1 Armor
\item It collects trinkets that humans would consider valuable (gold, gems, secrets): hoarder
\item It's from beyond this world: planar, write a move about using its otherworldly knowledge and power
\item It's kept alive by something beyond simple biology: +4 HP
\item It was made by someone: construct, give it a special quality or two about its construction or purpose
\item Its appearance is disturbing, terrible, or horrible: terrifying, write a special quality about why it's so horrendous
\item It doesn't have organs or discernible anatomy: amorphous, +1 Armor, +3 HP
\item It (or its species) is ancient--older than man, elves, and dwarves: increase its damage die size by one
\item It abhors violence: roll damage twice and take the worst result

\end{itemize}
\section*{Treasure}


 Monsters, much like adventurers, collect shiny useful things. When the players search the belongings of a monster (be they on their person or tucked away somewhere) describe them honestly.


 If the monster has accumulated some wealth you can roll that randomly. Start with the monster's damage die, modified if the monster is:
\begin{itemize}
\item Hoarder: roll damage die twice, take higher result
\item Far from home: add at least one ration (usable by anyone with similar taste)
\item Magical: some strange item, possibly magical
\item Divine: a sign of a deity (or deities)
\item Planar: something not of this earth
\item Lord over others: +1d4 to the roll
\item Ancient and noteworthy: +1d4 to the roll

\end{itemize}


 Roll the monster's damage die plus any added dice to find the monster's treasure:
\begin{enumerate}
\item A few coins, 2d8 or so
\item An item useful to the current situation
\item Several coins, about 4d10
\item A small item (gem, art) of considerable value, worth as much as 2d10\~A�10 coins, 0 weight
\item Some minor magical trinket
\item Useful information (in the form of clues, notes, etc.)
\item A bag of coins, 1d4\~A�100 or thereabouts. 1 weight per 100.
\item A very valuable small item (gem, art) worth 2d6\~A�100, 0 weight
\item A chest of coins and other small valuables. 1 weight but worth 3d6\~A�100 coins.
\item A magical item or magical effect
\item Many bags of coins for a total of 2d4\~A�100 or so
\item A sign of office (crown, banner) worth at least 3d4\~A�100 coins
\item A large art item worth 4d4\~A�100 coins, 1 weight
\item A unique item worth at least 5d4\~A�100 coins
\item All the information needed to learn a new spell and roll again
\item A portal or secret path (or directions to one) and roll again
\item Something relating to one of the characters and roll again
\item A hoard: 1d10\~A�1000 coins and 1d10\~A�10 gems worth 2d6\~A�100 each

\end{enumerate}
\section*{Monster Settings}


 The monsters in this book are presented in \emph{monster settings}. A monster setting is a type of location and the monsters you might find there. It's a way of grouping monsters by where they fit in the world. A monster setting tells you what kind of monsters might inhabit an area while your fronts tell you what monsters are working together or have ongoing plots.


 When creating your own monster settings, they can be more specific. You could create a monster setting for the Great Western Steppes or the Domains of the Horse Lords.


 Consult a monster setting to populate a front or when you want a threat that is only tangentially related to one of your fronts. For example, if the heroes are battling against the dungeon front, the cult of Khul-ka-ra, by exploring the ancient ruins that the cult has made its home then you might use monsters from the Legions of the Undead as a related threat--not truly part of the front but still a block in the heroes' path.


 The monster stat blocks within the settings describe HP, damage, and all the other aspects of the monster. These monsters were created with the same process listed above, and the reasons for their stats are just as important as the stats themselves. Looking at the reasoning behind the stats will allow you to present the monsters honestly, answering questions that arise in Dungeon World like ``can a warband of gnolls sack an entire village?''
\subsection{Cavern Dwellers}


 At the edges of civilization in the caves and tunnels below the old mountains of the world dwell all sorts of scheming, dangerous monsters. Some are wily and old, like the race of goblins scheming to burn villages and make off with livestock. Others are strange aberrations of nature like the stinking, trash-eating Otyugh. A word of caution, then, to those brave adventurers whose first foray into danger leads them into these dank and shadowy places; bad things live in the dark. Bad things with sharp teeth.
\subsection{Denizens of the Swamp}


 All things give way to rot in the end. Food spoils on the table, men's minds go mad with age and disease. Even the world itself, when left untended and uncared for, can turn to black muck and stinking air. Things dwell in these parts of Dungeon World. Things gone just as a bad as the swirling filth that fills the swamps. In these cesspit lowlands, adventurers will find such creatures as the deadly-eyed basilisk or the famed, unkillable troll. You'll need more than a dry pair of boots to survive these putrid fens. A sword would be a good start.
\subsection{Legions of the Undead}


 The sermons of mannish and dwarven gods would tell you that Death is the end of all. They say that once the mortal coil is unwound and a person takes their final breath that all is warmth and song and the white wings of angels. Not so. Not for all. For some, after life's embrace loses its strength a darker power can take hold. Black magic rips the dead from the ground and gives them shambling unlife full of hate and hunger. Sorcery and witchcraft lend an ancient spell-smith the power to live forever in the husk of a Lich. There are bleak enchantments at play in shadowy corners all throughout Dungeon World. These creatures are the spawn of that fell magic.
\subsection{The Dark Woods}


 It would not be a lie to say that there are trees that stand in the deepest groves of Dungeon World that have stood since before man or elf walked amidst their roots. It would be true, too, to say that these ancient trees have long lost the green leaves of spring. In the strands of the dark woods one finds, if one looks in the right place, sylvan monsters old and powerful. Here live the race of savage centaurs and the fey soul-stealing creatures of yore. Under the shadow of the ancient trees, wolf-men howl for blood. Hurry along the old forest road and light no fire for food or warmth for it's said that flames offend the woods themselves. You wouldn't want that, would you?
\subsection{Ravenous Hordes}


 ``I've bested an orc in single combat,'' they crow. ``I've fought a gnoll and lived to the tell the tale.'' Which is no small feat and yet, you know the truth of these boasts. Like vermin, spotting but one of these creatures speaks to a greater doom on the horizon. No orc travels alone. No slavering gnoll moves without his pack. You know that soon, the wardrums will sound and the walls will be besieged by the full fury of the warchief and his tusked berserkers. These are the monsters that will bring civilization, screaming and weeping, to its knees. Unless you can stop them. Best of luck.
\subsection{Twisted Experiments}


 For some who learn the arcane arts it's not merely enough to be able to live for a thousand years or throw lightning bolts that can fry a man. Some aren't quite satisfied with the power to speak to the dead or draw the angels down from heaven. Hubris calls on those cloaked-and-hooded ``scientists'' to make a strange and unholy life of their own. No mortal children, these. These are the brood of a mind gone foul with strange magic. In this setting you will find such nightmares as the chimera, dripping poison. Here, too, are the protector golems and mutant apes. All sorts of bad ideas await you in the fallen towers of the mad magicians of Dungeon World.
\subsection{The Lower Depths}


 Ruins dot the countryside of Dungeon World. Old bastions of long-forgotten civilization fallen to decay, to monsters, or to the whim of a vengeful god. These ruins often cover a much more dangerous truth--catacombs and underground complexes lousy with traps and monsters. Gold, too. Which is why you're here. Why you're locked in mortal combat with a tribe of spiteful dark elves. Battling stone giants in caverns the size of whole countries. Maybe, though, you're the noble souls who've travelled to the world's heart to put an end to the Apocalypse Dragon--the beast who, it is said, will one day swallow the sun and kill us all. We appreciate it, really. We'll all pray for you.
\subsection{Planar Powers}


 Sometimes, monsters do not come from Dungeon World at all. Beyond the mountains at the edge of the world or below the deepest seas, the sages and wise old priests say that there are gateways to the lands beyond. They speak of elysian fields; rivers of sweet wine and maidens dancing in fields of gold. They tell tales of the paradise of heavens to be found past the Planar Door. Tales tell, too, of the Thousandfold Hell. Of the swirling Elemental Vortex and the devils that wait for the stars to align so they can enter Dungeon World and wreak their bloody havoc. You must be curious to know if these tales are true? What will you see when the passage to the beyond is opened?


